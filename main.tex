% !TeX program = xelatex

\documentclass[11pt]{report}
\usepackage[scheme=plain]{ctex}
\usepackage{amsmath}
\usepackage{amssymb}
\usepackage{amsthm}
\usepackage{bm}
\usepackage{mathrsfs}
\usepackage{bbm}
\usepackage{xcolor}
\usepackage[shortlabels]{enumitem}
\usepackage{geometry}
\usepackage{siunitx}
\usepackage[unicode=true,colorlinks=true,linkcolor=red]{hyperref}
\usepackage{fixdif}
\usepackage{graphicx}

\setCJKmainfont{FandolSong}
\setlist{nosep}

\newcommand\diff{\mathop{}\!\mathrm{d}}
\newcommand{\e}{\mathrm{e}}
\newcommand{\upi}{\mathrm{i}}
\newcommand{\closure}[1]{\overline{#1}}
\DeclareMathOperator{\diag}{diag}
\DeclareMathOperator{\dist}{dist}

\setlist[enumerate,1]{left=\parindent}

\theoremstyle{remark}
\newtheorem*{remark}{注}
\newtheoremstyle{PDE}{3pt}{3pt}{\rmfamily}{\parindent}{\bfseries}{}{.5em}{}
\theoremstyle{PDE}
\newtheorem{note}{Note}[section]
\newtheorem{property}{性质}

\makeatletter
\renewenvironment{proof}{\par
	\pushQED{\qed}%
	\normalfont \topsep1\p@\@plus6\p@\relax
	\trivlist
	\item\relax
	{\hspace*{\parindent}{\itshape\bfseries Proof}\@addpunct{.}}\hspace\labelsep\ignorespaces
}{%
	\popQED\endtrivlist\@endpefalse
}
\newenvironment{solve}{\par
	\pushQED{\qed}%
	\normalfont \topsep1\p@\@plus6\p@\relax
	\trivlist
	\item\relax
	{\hspace*{\parindent}{\itshape\bfseries Solution}\@addpunct{.}}\hspace\labelsep\ignorespaces
}{%
	\popQED\endtrivlist\@endpefalse
}


\newenvironment{solution}{\par
	\pushQED{\qed}%
	\normalfont \topsep1\p@\@plus6\p@\relax
	\trivlist
	\item\relax
	{\hspace*{\parindent}{\itshape\bfseries Solution}\@addpunct{.}}\hspace\labelsep\ignorespaces
}{%
	\popQED\endtrivlist\@endpefalse
}
\makeatother
\newcounter{exercise}[section]
\NewDocumentEnvironment{exercise}{o d<> +b}
  {%
    \IfNoValueTF{#1}
      {\stepcounter{exercise}}
      {\setcounter{exercise}{#1}}
    \par\textbf{\theexercise.}\hspace{.333em}%
	\IfNoValueTF{#2}
	  {#3}
	  {\textbf{(#2)}\hspace{.333em}#3}
  }
  {\ignorespacesafterend}
\renewcommand{\div}{\operatorname{div}}
\let\leq=\leqslant
\let\geq=\geqslant

\title{\textbf{数学物理方程习题解答}}

\begin{document}

\maketitle
\tableofcontents

\input{chapter1.tex}
\chapter{热传导方程}

\section{热传导方程及其定解问题的导出}

\section{初边值问题的分离变量法}


\begin{exercise}
  用分离变量法求下列定解问题的解:
  \[\begin{cases}
    \frac{\partial u}{\partial t} = a^2 \frac{\partial^2u}{\partial x^2}
      \quad (t>0, 0<x<\pi), \\
    u(0,t) = \frac{\partial u}{\partial x}(\pi,t) = 0\quad (t>0), \\
    u(x,0) = f(x)\quad (0<x<\pi).
  \end{cases}\]
\end{exercise}

\begin{solution}
  利用分离变量法, 设 $u(x,t)=X(x)T(t)$, 则
  \[X(x)T'(t) = a^2X''(x)T(t).\]
  由此得
  \begin{align}
    X''(x)+\lambda X(x) & = 0, \label{eq:c1} \\
    T'(t)+a^2\lambda T(t) & = 0. \label{eq:c2} 
  \end{align}
  由 \eqref{eq:c1} 及 $X(x)$ 满足的边界条件 $X(0)=0$, $X'(\pi)=0$, 得
  \begin{enumerate}[(i)]
    \item 当 $\lambda\leq 0$ 时只有零解;
    \item 当 $\lambda>0$ 时, $X(x)=A\cos\sqrt{\lambda}x+B\sin\sqrt{\lambda}x$,
      代入边界条件得固有值为 $\lambda_k=\left(k+\frac{1}{2}\right)^2$,
      相应的固有函数 $X_k(x)=B_k\sin\sqrt{\lambda_k}x=B_k\sin\left(k+\frac{1}{2}\right)x$.
  \end{enumerate}
  将 $\lambda=\lambda_k$ 代入 \eqref{eq:c2} 得 $T_k(t)=C_ke^{-a^2\lambda_kt}$, 故
  \[u_k(x,t) = A_ke^{-a^2\lambda_kt}\sin\left(k+\frac{1}{2}\right)x,u(x,t)
    = \sum_{k=1}^{\infty}u_k(x,t).\]
  利用初始条件得
  \[f(x) = \sum_{k=1}^{\infty}A_k\sin\left(k+\frac{1}{2}\right)x\Rightarrow A_k
    = \frac{2}{\pi}\int_0^{\pi}f(\xi)\sin\left(k+\frac{1}{2}\right)\xi\diff\xi.\]
  因此原问题的解为
  \[u(x,t) = \sum_{k=1}^{\infty}\frac{2}{\pi}\int_0^{\pi} f(\xi)\sin\left(k+\frac{1}{2}\right)
    \xi\diff\xi\cdot e^{-a^2\lambda_kt}\sin\left(k+\frac{1}{2}\right)x. \qedhere\]
\end{solution}


\begin{exercise}
  用分离变量法求解热传导方程的初边值问题:
  \[\begin{cases}
    \frac{\partial u}{\partial t} = \frac{\partial^2 u}{\partial x^2} \quad (t>0,\, 0<x<1), \\
    u(x,0) = \begin{cases}
               x,     & 0 < x \leq\frac12, \\
               1 - x, & \frac12 < x < 1,
             \end{cases} \\
    u(0,t) = u(1,t) = 0 \quad (t>0).
  \end{cases}\]
\end{exercise}

\begin{solution}
  Suppose $u(x,t) = X(x)T(t)$, then
  \[ \frac{X''(x)}{X(x)} = \frac{T'(t)}{T(t)} = -\lambda. \]
  First of all, the function $X$ satisfies the following condition
  \begin{equation}\label{eq:c3}
    \begin{cases}
      X''(x) + \lambda X(x) = 0, \\
      X(0) = X(1) = 0.
    \end{cases}
  \end{equation}
  We can solve that the eigenvalues are $\lambda_k = (k\pi)^2$ and $X_k(x) = A_k\sin k\pi x$.

  Then $T(t)$ satisfies the following equation
  \[ T_k(t) + (k\pi)^2T_k(t) = 0, \]
  from which we can solve that $T_k(t) = B_k e^{-(k\pi)^2t}$. Hence
  we could write $u(x,t)$ as
  \begin{equation}\label{eq:c4}
    u(x,t) = \sum_{k=1}^\infty C_k e^{-(k\pi)^2t} \sin k\pi x.
  \end{equation}
  Using the initial value condition we find that
  \[ \sum_{k=1}^\infty C_k \sin k\pi x = f(x) = \begin{cases}
    x,     & 0 < x \leq\frac12, \\
    1 - x, & \frac12 < x < 1.
  \end{cases} \]
  Hence
  \begin{equation}\label{eq:c5}
    C_k = 2 \int_0^1 f(x) \sin k\pi x \d x = \frac{4}{k^2\pi^2} \sin\frac{k\pi}{2}.
  \end{equation}
  From \eqref{eq:c4} and \eqref{eq:c5} we get
  \[ u(x,t) = \sum_{k=1}^\infty \frac{4}{k^2\pi^2} \sin\frac{k\pi}{2}
  e^{-(k\pi)^2t} \sin k\pi x. \qedhere \]
\end{solution}

\begin{exercise}
  如果有一根长度为 $l$ 的均匀细棒, 其周围以及两端 $x=0$, $x=l$
  均为绝热, 初始温度分布为 $u(x,0)=f(x)$, 问以后的温度分布如何?
  且证明当 $f(x)$ 等于常数 $u_0$ 时, 恒有 $u(x,t)=u_0$.
\end{exercise}

\begin{solution}
  因为细棒的两端均为绝热,故根据傅里叶定律知$u_x|_{x=0}=u_x|_{x=l}=0$,此初边值问题为
  \[\begin{cases}
    u_t = a^2u_{xx}, \\
    u_x|_{x=0} = u_x|_{x=l} = 0, \\
    u|_{t=0} = f(x).
  \end{cases}\]
  直接解得
  \[ u(x,t)=\sum_{k=0}^{\infty}D_ke^{-a^2\lambda_kt}\cos\frac{k\pi}{l}x,
    \quad \lambda_k=\left(\frac{k\pi}{l}\right)^2, \]
  其中
  \[ D_0 = \frac{1}{l}\int_0^lf(\xi) \d\xi,
    \quad D_k=\frac{2}{l}\int_0^lf(\xi)\cos\frac{k\pi}{l}\xi \d\xi
    \quad (k\geq1).\]
  当 $f(x)\equiv u_0$ 时, $D_0=u_0$, $D_k=0$ $(k=1,2,\cdots)$, 故 $ u(x,t)=u_0$.
\end{solution}


\begin{exercise}
  在区域 $t>0$, $0<x<l$ 中求解如下的定解问题:
  \[\begin{cases}
    \frac{\partial u}{\partial t} = a^2 \frac{\partial^2u}{\partial x^2} - \beta(u-u_0), \\
    u(0,t) = u(l,t) = u_0, \\
    u(x,0) = f(x),
  \end{cases}\]
  其中 $a$, $\beta$, $u_0$ 均为常数, $f(x)$ 为已知函数.
\end{exercise}

\begin{solution}
  作变量代换,令 $v(x,t)=(u-u_0)e^{\beta t}$,则$v(x,t)$满足的定解问题为:
  \[\begin{cases}
    v_t=a^2v_{xx}, \\
    v(0,t)=v(l,t)=0, \\
    v(x,0)=f(x)-u_0.
  \end{cases}\]
  直接解得
  \[ v(x,t) = \sum_{k=1}^{\infty}A_ke^{-a^2\lambda_kt}\sin\frac{k\pi}{l}x,
    \quad \lambda_k = \left(\frac{k\pi}{l}\right)^2, \]
  其中
  \[ A_k = \frac{2}{l}\int_0^l(f(\xi)-u_0)\sin\frac{k\pi}{l}\xi\diff\xi. \]
  因此
  \[u(x,t) = u_0 + \sum_{k=1}^{\infty}\frac{2}{l}
    \int_0^l(f(\xi)-u_0)\sin\frac{k\pi}{l}\xi\diff\xi
      \cdot e^{-\left(\frac{a^2k^2\pi^2}{l^2}+\beta\right)t}\sin\frac{k\pi}{l}x. \qedhere \]
\end{solution}


\begin{exercise}
  长度为 $l$ 的均匀细杆的初始温度为 \qty{0}{\degreeCelsius}, 端点 $x=0$ 保持恒温 $u_0$,
  而在 $x=l$ 和侧面上, 热量可以发散到周围的介质中去, 介质的温度为 \qty{0}{\degreeCelsius},
  此时杆上的温度分布函数 $u(x,t)$ 满足下述定解问题:
  \[\begin{cases}
    \frac{\partial u}{\partial t} = a^2 \frac{\partial^2 u}{\partial x^2} - b^2 u, \\
    u(0,t) = u_0, \quad \Bigl(\frac{\partial u}{\partial x} + Hu\Bigr)\Big|_{x=l} = 0, \\
    u(x,0) = 0,
  \end{cases}\]
  其中 $a$, $b$, $H$ 均为常数, 试求出 $u(x,t)$.
\end{exercise}

\begin{solution}
    Let $u(x,t) = u_1(x,t) + u_2(x)$ in which $u_1$ satisfies the following 
    problem with homogeneous boundary condition
    \begin{equation}\label{eq:c6}
      \begin{cases}
        \frac{\partial u_1}{\partial t} = a^2 \frac{\partial^2 u_1}{\partial x^2} - b^2u_1, \\
        u_1(0,t) = \bigl(\frac{\partial u_1}{\partial x} + Hu_1\bigr)\big|_{x=l} = 0, \\
        u_1(x,0) = -u_2,
      \end{cases}
    \end{equation}
    and $u_2$ satisfies the following ordinary differential equation
    \begin{equation}\label{eq:c7}
      \begin{cases}
        a^2 \frac{\d^2 u_2}{\d x^2} - b^2 u_2 = 0, \\
        u_2(0) = u_0, \\
        \bigl(\frac{\d u_2}{\d x} + Hu_2\bigr)\big|_{x=l} = 0.
      \end{cases}
    \end{equation}
    We first solve equation \eqref{eq:c7}. Since $a^2 \frac{\d^2 u_2}{\d x^2} - b^2 u_2 = 0$,
    we have that
    \begin{equation}\label{eq:c8}
      u_2 = C_1 e^{\frac{b}{a}x} + C_2 e^{-\frac{b}{a}x}.
    \end{equation}
    By the boundary condition we get
    \[\begin{cases}
      C_1 + C_2 = 0, \\
      C_1\frac{b}{a}e^{\frac{b}{a}l} - C_2 \frac{b}{a} e^{-\frac{b}{a}l}
        + H C_1 e^{\frac{b}{a}l} + H C_2 e^{-\frac{b}{a}l} = 0.
    \end{cases}\]
    Solving $C_1$, $C_2$ and substituting them into \eqref{eq:c8}, we have
    \begin{equation}\label{eq:c9}
      u_2 = u_0 \cosh \frac{b}{a}x - \frac{H\cosh\frac{b}{a}l + \frac{b}{a}\sinh\frac{b}{a}l}
        {H\sinh\frac{b}{a}l + \frac{b}{a}\cosh\frac{b}{a}l}
        u_0 \sinh\frac{b}{a}x.
    \end{equation}

    Now we solve equation \eqref{eq:c6}. Let $v = e^{b^2 t}u_1$, then $v$ satisfies
    \begin{equation}\label{eq:c10}
      \begin{cases}
        \frac{\partial v}{\partial t} = a^2 \frac{\partial^2 v}{\partial x^2}, \\
        v(0,t) = \bigl(\frac{\partial v}{\partial x} + Hv\bigr)\big|_{x=l} = 0, \\
        v(x,0) = -u_2.
      \end{cases}      
    \end{equation}
    The procedure of solving this problem is actually the same as that in the textbook
    from Page 51 to 54.
    Denote $v(x,t) = X(x)T(t)$, then
    \[ \frac{X''(x)}{X(x)} = \frac{T'(t)}{a^2 T(t)} = -\lambda \]
    for some constant $\lambda$. First of all, $X(x)$ satisfies
    \begin{equation}\label{eq:c11}
      \begin{cases}
        X''(x) + \lambda X(x) = 0, \\
        X(0) = X'(l) + HX(l) = 0.
      \end{cases}
    \end{equation}
    \begin{itemize}
      \item If $\lambda\leq 0$, there only exists trivial solution $X\equiv 0$;
      \item If $\lambda>0$,
        \begin{equation}\label{eq:c12}
          X(x) = A\cos\sqrt{\lambda}x + B\sin\sqrt{\lambda}x.
        \end{equation}
        Combining with the boundary conditions we have
        \begin{equation}\label{eq:c13}
          X_k(x) = B_k \sin\sqrt{\lambda_k} x,
        \end{equation}
        where $(\lambda_k)_{k\geq 1}$ is the sequence of positive solutions to
        $\sqrt\lambda + H\tan\sqrt\lambda l = 0$.
    \end{itemize}

    On the other hand, $T(t)$ satisfies
    \begin{equation}\label{eq:c14}
      T'(t) + a^2\lambda_k T(t) = 0,
    \end{equation}
    from which we get
    \begin{equation}\label{eq:c15}
      T_k(t) = C_k e^{-a^2\lambda_k t}.
    \end{equation}
    Therefore according to \eqref{eq:c13} and \eqref{eq:c15} we can write $v(x,t)$ as
    \begin{equation}\label{eq:c16}
      v(x,t) = \sum_{k=1}^\infty A_k e^{-a^2\lambda_k t} \sin\sqrt{\lambda_k} x.
    \end{equation}
    Finally we need to utilize the initial value condition to get
    \begin{equation}\label{eq:c17}
      v(x,0) = \sum_{k=1}^\infty A_k \sin\sqrt{\lambda_k} x = -u_0.
    \end{equation}
    Since
    \begin{equation}\label{eq:c18}
      \bigl(\sin\sqrt{\lambda_m}x, \sin\sqrt{\lambda_n}x\bigr)_{L^2}
       = \delta_{mn} \biggl(\frac{l}{2} + \frac{H}{2(H^2+\lambda_m)}\biggr)
       =: \delta_{mn} \alpha_m,
    \end{equation}
    we have
    \begin{equation}\label{eq:c19}
      A_k = -\frac{1}{\alpha_k} \int_0^l u_0(x) \sin\sqrt{\lambda_k} x \d x.
    \end{equation}
    Hence
    \begin{equation}\label{eq:c20}
      v(x,t) = -\sum_{k=1}^\infty \frac{1}{\alpha_k}
        \int_0^l u_0(x) \sin\sqrt{\lambda_k}x \d x \cdot
        e^{-a^2\lambda_k t} \sin\sqrt{\lambda_k} x
    \end{equation}
    and
    \begin{equation}\label{eq:c21}
      u_1(x,t) = -\sum_{k=1}^\infty \frac{1}{\alpha_k}
        \int_0^l u_0(x) \sin\sqrt{\lambda_k}x \d x \cdot
        e^{-(a^2\lambda_k + b^2)t} \sin\sqrt{\lambda_k} x
    \end{equation}
    Finally we conclude that
    \begin{equation}\label{eq:c22}
      u(x,t) = u_1(x,t) + u_2(x),
    \end{equation}
    where $u_1$ and $u_2$ are given by \eqref{eq:c21} and \eqref{eq:c9}
    respectively.
\end{solution}


\section{柯西问题}

\begin{exercise}
  求下列函数的 Fourier 变换:
  \begin{enumerate}[(1),left=\parindent]
    \item $\e^{-\eta x^2}$ $(\eta>0)$;
    \item $\e^{-a|x|}$ $(a>0)$;
    \item $\frac{x}{(a^2+x^2)^k}$, $\frac{1}{(a^2+x^2)^k}$ ($a>0$, $k$ 为自然数).
  \end{enumerate}
\end{exercise}

\begin{solution}
  (1) 直接计算得
  \begin{align*}
    \widehat{\e^{-\eta x^2}}(\xi)
    & = \int_{-\infty}^{\infty} \e^{-\eta x^2}\cdot \e^{-\upi x\xi}\diff x
      = \e^{-\frac{\xi^2}{4\eta}} \int_{-\infty}^{\infty}
      \e^{-\eta\left(\xi+\frac{\upi\xi}{2\eta}\right)^2} \diff x \\
    & = \e^{-\frac{\xi^2}{4\eta}} \int_{-\infty}^{\infty}
      \e^{-y^2} \frac{1}{\sqrt{\eta}} \diff y
      = \biggl(\frac{\pi}{\eta}\biggr)^{1/2} \e^{-\frac{\xi^2}{4\eta}}.
    \end{align*}

  (2) 直接计算得
  \begin{align*}
    \widehat{\e^{-a|x|}} (\xi)
    & = \int_{-\infty}^{\infty} \e^{-a|x|}\cdot\e^{-\upi x\xi} \diff x \\
    & = 2\int_0^{\infty} \e^{-ax} \cos\xi x \diff x = \frac{2a}{a^2+\xi^2}.
  \end{align*}
\end{solution}


\begin{exercise}
  证明: 当 $f(x)$ 在 $(-\infty,\infty)$ 上绝对可积时, $F[f]$ 为连续函数.
\end{exercise}

\begin{proof}
  记 $F[f]=\int_{-\infty}^{\infty} f(\xi)\e^{-\upi\lambda\xi}\diff\xi = \tilde{f}(\lambda)$,则
  \[\begin{split}
    \bigl|\tilde{f}(\lambda+h)-\tilde{f}(\lambda)\bigr|
    & = \left|\int_{-\infty}^{\infty}f(\xi)
        \left(\e^{-\upi(\lambda+h)\xi} - \e^{-\upi\lambda\xi}\right)\diff\xi\right| \\
    & \leq \int_{-\infty}^{\infty}|f(\xi)|\cdot|\e^{-\upi h\xi}-1|\diff\xi\to 0
      \quad(\text{as } h\to 0),
  \end{split}\]
  故 $F[f]$ 为连续函数.
\end{proof}


\begin{exercise}[4]
  证明 (3.29) 所表示的函数满足非齐次方程 (3.15) 以及初始条件 (3.16).
\end{exercise}

\begin{proof}
  同教材上面验证齐次方程的情形.
\end{proof}


\begin{exercise}[5]
  求解热传导方程 (3.17) 的柯西问题, 已知
  \begin{enumerate}[(1)]
    \item $u|_{t=0} = \sin x$,
    \item 用延拓法求解半有界直线上的热传导方程 (3.17), 假设
      \[\begin{cases}
        u(x,0) = \varphi(x)\quad (0<x<\infty), \\
        u(0,t) = 0.
      \end{cases}\]
  \end{enumerate}
\end{exercise}

\begin{solution}
  由泊松公式知
  \[\begin{split}
    u(x,t)
    & = \frac{1}{2a\sqrt{\pi t}} \int_{-\infty}^{\infty}
        \sin\xi\cdot \e^{-\frac{(x-\xi)^2}{4a^2t}}\diff\xi \\
    & = \frac{1}{2a\sqrt{\pi t}}\cdot 2a\sqrt{t}\int_{-\infty}^{\infty}
        \sin(x-2a\sqrt{t}\zeta)\cdot \e^{-\zeta^2}\diff\zeta \\
    & = \frac{1}{\sqrt{\pi}} \int_{-\infty}^{\infty}
        \sin x\cos 2a\sqrt{t}\zeta\cdot e^{-\zeta^2}\diff\zeta \\
    & = \frac{2\sin x}{\sqrt{\pi}} \int_0^{\infty}
        \cos 2a\sqrt{t}\zeta\cdot e^{-\zeta^2}\diff\zeta \\
    & = \e^{-a^2t}\sin x.
  \end{split}\]

  (2) 对 $\varphi(x)$ 作奇延拓, 即
  \[\varPhi(x) = \begin{cases}
    \varphi(x),   & x\geq 0, \\
    -\varphi(-x), & x<0.
  \end{cases}\]
  求解如下 Cauchy 问题
  \[\begin{cases}
    u_t = a^2 u_{xx}, \\
    u|_{t=0} = \varPhi(x).
  \end{cases}\]
  得
  \begin{align*}
    u(x,t)
    & = \frac{1}{2a\sqrt{\pi t}} \int_{-\infty}^{\infty}
        \varPhi(\xi) \e^{-\frac{(x-\xi)^2}{4a^2t}} \diff\xi \\
    & = \frac{1}{2a\sqrt{\pi t}} \biggl[
        \int_0^{\infty} \varphi(\xi) \e^{-\frac{(x-\xi)^2}{4a^2t}} \diff\xi
        + \int_{-\infty}^0 -\varphi(-\xi) \e^{-\frac{(x-\xi)^2}{4a^2t}}\diff\xi\biggr] \\
    & = \frac{1}{a\sqrt{\pi t}} \int_0^{\infty} \varphi(\xi)
        \e^{-\frac{x^2+\xi^2}{4a^2 t}}\sinh \frac{x\xi}{2a^2 t}\diff\xi. \qedhere
  \end{align*}
\end{solution}


\begin{exercise}[7]
  证明: 如果 $u_1(x,t)$, $u_2(y,t)$ 分别是下述两个定解问题的解:
  \[\begin{cases}
    \frac{\partial u_1}{\partial t} = a^2 \frac{\partial^2u_1}{\partial x^2}, \\
    u_1|_{t=0} = \varphi_1(x);
  \end{cases}
  \qquad
  \begin{cases}
    \frac{\partial u_2}{\partial t} = a^2 \frac{\partial^2u_2}{\partial y^2}, \\
    u_2|_{t=0} = \varphi_2(y).
  \end{cases}\]
  则 $u(x,y,t) = u_1(x,t)u_2(y,t)$ 是定解问题
  \[\begin{cases}
    \frac{\partial u}{\partial t} = a^2\biggl(
      \frac{\partial^2u}{\partial x^2} + \frac{\partial^2u}{\partial y^2}\biggr), \\
    u|_{t=0} = \varphi_1(x) \varphi_2(y)
  \end{cases}\]
  的解.
\end{exercise}

\begin{solution}
  直接验证.
\end{solution}


\begin{exercise}[8]
  导出下列热传导方程柯西问题解的表达式:
  \[\begin{cases}
    \frac{\partial u}{\partial t} 
      = a^2\biggl(\frac{\partial^2u}{\partial x^2} +
                  \frac{\partial^2u}{\partial y^2}\biggr), \\
    u|_{t=0} = \sum_{i=1}^n \alpha_i(x)\beta_i(y).
  \end{cases}\]
\end{exercise}

\begin{solution}
  由叠加原理与上题结果或直接应用 Fourier 变换可得解为
  \[u(x,y,t) = \frac{1}{4a^2\pi t} \sum_{i=1}^n \int_{-\infty}^{\infty}
    \int_{-\infty}^{\infty} \alpha_i(\xi) \beta_i(\eta)
      \exp\biggl(-\frac{(x-\xi)^2+(y-\eta)^2}{4a^2 t}\biggr) \diff\xi\diff\eta.\qedhere\]
\end{solution}


\begin{exercise}[9]
  验证二维热传导方程柯西问题
  \[\begin{cases}
    \frac{\partial u}{\partial t} = a^2 
      \Bigl(\frac{\partial^2 u}{\partial x^2} + \frac{\partial^2 u}{\partial y^2}\Bigr), \\
    u|_{t=0} = \varphi(x,y)
  \end{cases}\]
  的解的表达式为
  \[u(x,y,t) = \frac{1}{4\pi a^2t} \int_{-\infty}^{\infty}
    \int_{-\infty}^{\infty} \varphi(\xi,\eta) \e^{-\frac{(x-\xi)^2+(y-\eta)^2}{4a^2t}}
    \diff\xi\diff\eta.\]
\end{exercise}

\begin{proof}
  本习题应该添加假设: $\varphi(x,y)$有界, 因为
  \[u(x,y,t) = \frac{1}{4\pi a^2t}\int_{-\infty}^{\infty}
    \int_{-\infty}^{\infty}\varphi(\xi,\eta)
      \e^{-\frac{(x-\xi)^2+(y-\eta)^2}{4a^2t}}\diff\xi\diff\eta,\]
  所以
  \[\begin{split}
    \frac{\partial u}{\partial t}
    ={} & \frac{-1}{4\pi a^2t^2}\int_{-\infty}^{\infty}\int_{-\infty}^{\infty}
      \varphi(\xi,\eta) \e^{-\frac{(x-\xi)^2+(y-\eta)^2}{4a^2t}}\diff\xi\diff\eta \\
        & + \frac{1}{4\pi a^2t}\int_{-\infty}^{\infty}\int_{-\infty}^{\infty}
          \varphi(\xi,\eta) \e^{-\frac{(x-\xi)^2+(y-\eta)^2}{4a^2t}}
          \cdot\frac{(x-\xi)^2+(y-\eta)^2}{4a^2t^2}\diff\xi\diff\eta.
  \end{split}\]
  又
  \[\frac{\partial u}{\partial x} = 
    \frac{1}{4\pi a^2t}\int_{-\infty}^{\infty}\int_{-\infty}^{\infty}
    \varphi(\xi,\eta) \e^{-\frac{(x-\xi)^2+(y-\eta)^2}{4a^2t}}
    \cdot\frac{-(x-\xi)}{2a^2t}\diff\xi\diff\eta.\]
  故
  \[\frac{\partial^2u}{\partial x^2} =
    \frac{1}{4\pi a^2t}\int_{-\infty}^{\infty}\int_{-\infty}^{\infty}
    \varphi(\xi,\eta) \e^{-\frac{(x-\xi)^2+(y-\eta)^2}{4a^2t}}
    \left(\frac{-1}{2a^2t} + \frac{(x-\xi)^2}{4a^4t^2}\right) \diff\xi\diff\eta.\]
  显然 $\frac{\partial^2u}{\partial y^2}$ 的结果形式同 $\frac{\partial^2u}{\partial x^2}$, 故
  \begin{align*}
    & a^2\left(\frac{\partial^2u}{\partial x^2}+\frac{\partial^2u}{\partial y^2}\right) \\
    ={} & \frac{1}{4\pi t}\int_{-\infty}^{\infty}\int_{-\infty}^{\infty}
          \varphi(\xi,\eta) e^{-\frac{(x-\xi)^2+(y-\eta)^2}{4a^2t}}
          \left(\frac{-1}{a^2t}+\frac{(x-\xi)^2+(y-\eta)^2}{4a^4t^2}\right)\diff\xi\diff\eta.
  \end{align*}
  对比可知
  \[\frac{\partial u}{\partial t}
    = a^2\left(\frac{\partial^2u}{\partial x^2}+\frac{\partial^2u}{\partial y^2}\right).\]
  
  对于初值的检验可对照教材P61的方法,下面不妨简单叙述一下.
  要证明当 $t\to 0$, $x\to x_0$, $y\to y_0$ 时,
  $u(x,y,t)\to\varphi(x_0,y_0)$, 令 $\zeta=\frac{x-\xi}{2a\sqrt{t}}$,
  $\theta=\frac{y-\eta}{2a\sqrt{t}}$,则
  \[u(x,y,t) =
    \frac{1}{\pi}\int_{-\infty}^{\infty}\int_{-\infty}^{\infty}
    \varphi(x-2a\sqrt{t}\zeta,y-2a\sqrt{t}\theta) \e^{-(\zeta^2+\theta^2)}
    \diff\zeta\diff\theta.\]
  而
  \[\varphi(x_0,y_0) =
    \frac{1}{\pi}\int_{-\infty}^{\infty}\int_{-\infty}^{\infty}
    \varphi(x_0,y_0) \e^{-(\zeta^2+\theta^2)} \diff\zeta\diff\theta.\]
  故
  \[u(x,y,t) - \varphi(x_0,y_0) =
    \frac{1}{\pi} \int_{-\infty}^{\infty}\int_{-\infty}^{\infty}
    \left[\varphi(x-2a\sqrt{t}\zeta,y-2a\sqrt{t}\theta)-\varphi(x_0,y_0)\right]
    \e^{-(\zeta^2+\theta^2)}\diff\zeta\diff\theta.\]
  将 $(\zeta,\theta)$ 平面用正方形 (四个顶点为$(\pm N,\pm N)$) 分成两个部分.
  在正方形内部, 利用$\varphi(x,y)$的连续性控制, 在正方形的外部,
  用积分$\int_{-\infty}^{\infty}\int_{-\infty}^{\infty}e^{-(\zeta^2+\theta^2)}\diff\zeta\diff\theta$ 可以任意小以及$\varphi(x,y)$是有界的来进行控制即可证明.
\end{proof}


\section{极值原理,定解问题解的唯一性和稳定性}

\begin{exercise}
  证明方程 $\frac{\partial u}{\partial t} = a^2 \frac{\partial^2 u}{\partial x^2} + cu$
  ($c\geq 0$) 具狄利克雷边界条件的初边值问题解的唯一性和稳定性.
\end{exercise}

\begin{proof}
  设$u(x,t)$满足的定解问题为
  \[\begin{cases}
    u_t = a^2 u_{xx} + cu, \\
    u(x,0) = \varphi(x), \\
    u(\alpha,t) = \mu_1(t),\,u(\beta,t)=\mu_2(t).
  \end{cases}\]
  则令 $v(x,t)=u(x,t) \e^{-ct}$, 可得 $v(x,t)$ 满足的定解问题为
  \[\begin{cases}
    v_t = a^2v_{xx}, \\
    v(x,0) = \varphi(x), \\
    v(\alpha,t) = \mu_1(t)\e^{-ct},\,v(\beta,t) = \mu_2(t)\e^{-ct}.
  \end{cases}\]
  由定理4.2知上述定解问题的解是唯一的且稳定的,
  记为 $v=v_0(x,t)$, 则原定解问题的解为$u = u_0(x,t) = \e^{ct}v_0(x,t)$, 显然也是唯一的且稳定的.
\end{proof}


\begin{exercise}
  利用热传导方程极值原理的方法, 证明二维调和函数在有界区域上的最大值不会超过
  它在边界上的最大值.
\end{exercise}

\begin{proof}
  记有界闭区域为 $\Omega$, 其边界为 $\Gamma$.
  设 $u(x,y)$ 在 $\Omega$ 上的最大值为 $M$, 在 $\Gamma$ 上的最大值为 $m$.
  假设在区域内部存在某点 $(x_0,y_0)$ 使得
  \[u(x_0,y_0) = M > m.\]
  作辅助函数
  \[V(x,y) = u(x,y)+\frac{M-m}{4R^2}\left[(x-x_0)^2+(y-y_0)^2\right].\]
  其中 $\Omega\subset B(0,R)$,我们有
  \[V(x_0,y_0)=u(x_0,y_0)=M.\]
  而在 $\Gamma$ 上
  \[V(x,y) < m+\frac{M-m}{4}=\theta M\quad (0<\theta<1).\]
  因此 $V(x,y)$ 必在区域 $\Omega$ 内部某点 $(x_1,y_1)$ 取得最大值,
  在这个点应有 $V_{xx}\leq0,V_{yy}\leq 0$, 但是
  \[V_{xx}+V_{yy}=u_{xx}+u_{yy}+\frac{M-m}{2R^2}>0.\]
  矛盾, 故成立 $M=m$.
\end{proof}


\begin{exercise}
  导出初边值问题
  \[\begin{cases}
    u_t - a^2 u_{xx} = f(x,t), \\
    u|_{x=0} = \mu_1(t),\quad
      \Bigl(\frac{\partial u}{\partial x} + hu\Bigr)\Bigm|_{x=l} = \mu_2(t)\quad (h>0), \\
    u|_{t=0} = \varphi(x)
  \end{cases}\]
  的解 $u(x,t)$ 在 $R_T=\{0\leq t\leq T, 0\leq x\leq l\}$ 中满足估计
  \[u(x,t) \leq \e^{\lambda T} \max\biggl\{
    0, \max_{0\leq x\leq l} \varphi(x),
    \max_{0\leq t\leq T} \biggl(\e^{-\lambda t}\mu_1(t),
      \frac{\e^{-\lambda t}\mu_2(t)}{h}\biggr),
      \frac{1}{\lambda} \max_{R_T} (\e^{-\lambda t}f)
  \biggr\},\]
  其中 $\lambda>0$ 为任意正常数.
\end{exercise}

\begin{proof}
  令 $v = \e^{-\lambda t}u$, 则 $v$ 满足的定解问题为
  \[\begin{cases}
    v_t-a^2v_{xx}+\lambda v = \e^{-\lambda t}f(x,t), \\
    v(x,0) = \varphi(x), \\
    v(0,t) = \e^{-\lambda t}\mu_1(t),\ (v_x+hv)|_{x=l} = \e^{-\lambda t}\mu_2(t).
  \end{cases}\]
  由极值原理知 $v$ 的极大值只能在边界 $x=0,x=l,t=0$上取到, 记极大值点为$(x_0,t_0)$.

  当 $(x_0,t_0)\in\{x=0\}\cup\{t=0\}$ 时,
  \[v(x_0,t_0) \leq \max\left(\max_{0\leq x\leq l} \varphi(x),
    \max_{0\leq t\leq T} \e^{-\lambda t}\mu_1(t)\right).\]

  当 $(x_0,t_0)\in\{x=l\}$ 时,
  由 $\frac{\partial v}{\partial x}\geq 0$ 知
  \[v(x_0,t_0)\leq\max_{0\leq t\leq T} \frac{1}{h} \e^{-\lambda t}\mu_2(t).\]
  故
  \[u(x,t)\leq \e^{\lambda t}\max\left(0,\max_{0\leq x\leq l}\varphi(x),
    \max_{0\leq t\leq T} \e^{-\lambda t}\mu_1(t),
    \max_{0\leq t\leq T} \frac{1}{h} \e^{-\lambda t}\mu_2(t)\right). \qedhere\]
\end{proof}


\section{解的渐近性态}

\begin{exercise}
  证明方程
  \[\begin{cases}
    u_t - a^2 u_{xx} = 0, \\
    u|_{x=0} = u|_{x=l} = 0, \\
    u|_{t=0} = \varphi(x)
  \end{cases}\]
  的解当 $t\to +\infty$ 时指数地衰减于零, 其中 $\varphi\in C^2$,
  且 $\varphi(0) = \varphi(l) = 0$.
\end{exercise}

\begin{proof}
  运用分离变量法求得定解问题的解为
  \[u(x,t) = \sum_{k=1}^{\infty} A_k \e^{-\frac{k^2\pi^2a^2}{l^2}t}\sin\frac{k\pi}{l}x.\]
  其中 $A_k=\frac{2}{l} \int_0^l \varphi(\xi) \sin\frac{k\pi}{l}\xi\diff\xi$,
  由 $\varphi$ 有界知 $\exists C_1>0$, 使得 $|A_k|\leq C_1$, 故当 $t>1$ 时,
  \[\begin{split}
  |u(x,t)|
  & \leq C_1\sum_{k=1}^{\infty} \e^{-\frac{k^2\pi^2a^2}{l^2}t} \\
  & = C_1\biggl(1+\sum_{k=2}^{\infty}e^{-\frac{(k^2-1)\pi^2a^2}{l^2}t}\biggr)
      \e^{-\frac{\pi^2a^2}{l^2}t} \\
  & \leq C_1\biggl(1+\sum_{k=2}^{\infty}e^{-\frac{(k^2-1)\pi^2a^2}{l^2}}\biggr)
      \e^{-\frac{\pi^2a^2}{l^2}t} \\
  & < C\e^{-\frac{\pi^2a^2}{l^2}t}.
  \end{split}\]
  因此解当 $t\to+\infty$ 时指数地衰减于零.
\end{proof}


\begin{exercise}
  证明: 当 $\varphi(x,y)$ 为 $\mathbb{R}^2$ 上的有界连续函数, 且 $\varphi\in L^1(\mathbb{R}^2)$
  时, 二维热传导方程柯西问题的解, 当 $t\to +\infty$ 时, 以 $t^{-1}$ 衰减率趋于零.
\end{exercise}

\begin{proof}
  \[\begin{split}
    |u(x,y,t)|
    & = \left|\frac{1}{4\pi a^2t} \iint_{\mathbb{R}^2} \varphi(\xi,\eta)
        \e^{-\frac{(x-\xi)^2+(y-\eta)^2}{4a^2t}}\diff\xi\diff\eta\right| \\
    & \leq \frac{1}{4\pi a^2t} \iint_{\mathbb{R}^2} |\varphi(x,y)|
        \e^{-\frac{(x-\xi)^2+(y-\eta)^2}{4a^2t}}\diff\xi\diff\eta \\
    & \leq Ct^{-1},
  \end{split}\]
  其中 $C$ 是仅与 $a$ 和 $\|\varphi\|_{L^1(\mathbb{R}^2)}$ 有关的正常数.
\end{proof}
\chapter{调和方程}

\section{建立方程,定解条件}

\begin{exercise}
  设 $u(x_1,\cdots,x_n)=f(r)$ (其中 $r=\sqrt{x_1^2+\cdots+x_n^2}$)
  是 $n$ 维调和函数 $\Bigl( \text{即满足方程\ }
  \frac{\partial^2u}{\partial x_1^2}+\cdots+\frac{\partial^2u}{\partial x_n^2}=0\Bigr)$,
  试证明
  \[f(r) = c_1+\frac{c_2}{r^{n-2}}\quad (n\geq 2),\]
  \[f(r) = c_1+c_2\ln\frac{1}{r}\quad (n=2),\]
  其中 $c_1$, $c_2$ 为任意常数.
\end{exercise}

\begin{proof}
  因为 $\frac{\partial u}{\partial x_i}=\frac{\diff f}{\diff r}\frac{x_i}{r}$,所以
  \[\frac{\partial^2u}{\partial x_i^2} =
    \frac{\diff^2f}{\diff r^2}\frac{x_i^2}{r^2}
    + \frac{\diff f}{\diff r}\frac{r^2-x_i^2}{r^3},\]
  故
  \[\sum_{i=1}^n\frac{\partial^2u}{\partial x_i^2}
    = \frac{\diff^2f}{\diff r^2}+\frac{\diff f}{\diff r}\frac{n-1}{r}
    = f''(r)+\frac{n-1}{r}f'(r)=0.\]
  由上式得
  \[f'=cr^{1-n}.\]
  \begin{enumerate}[(i)]
    \item $n = 2$ 时, $f(r)=c_1+c_2\ln r=c_1+c_2\ln\frac{1}{r}$,
    \item $n\neq 2$ 时, $f(r)=c_1+\frac{c_2}{2-n}r^{2-n}=c_1+\frac{c_2}{r^{n-2}}$,
  \end{enumerate}
  其中 $c_1$, $c_2$ 为任意常数.
\end{proof}


\begin{exercise}
  证明: 拉普拉斯算子在球坐标$(r,\theta,\varphi)$下可以写成
  \[\Delta u
    = \frac{1}{r^2}\frac{\partial}{\partial r}\left(r^2\frac{\partial u}{\partial r}\right)
    + \frac{1}{r^2\sin\theta}\frac{\partial}{\partial\theta}
      \left(\sin\theta\frac{\partial u}{\partial\theta}\right)
    +\frac{1}{r^2\sin^2\theta}\frac{\partial^2u}{\partial\varphi^2}.\]
\end{exercise}

\begin{proof}
  球坐标变换及其逆变换为
  \[\begin{cases}
    x = r\sin\theta\cos\varphi, \\
    y = r\sin\theta\sin\varphi, \\
    z = r\cos\theta;
  \end{cases}\Rightarrow
  \begin{cases}
    r = \sqrt{x^2+y^2+z^2}, \\
    \theta = \arccos\frac{z}{\sqrt{x^2+y^2+z^2}}, \\
    \varphi = \arctan\frac{y}{x}.
  \end{cases}\]
  通过链式法则可得
  \[\frac{\partial u}{\partial x}
    = \frac{\partial u}{\partial r}\frac{\partial r}{\partial x}
      + \frac{\partial u}{\partial\theta}\frac{\partial\theta}{\partial x}
      + \frac{\partial u}{\partial\varphi}\frac{\partial\varphi}{\partial x}.\]
  \[\frac{\partial^2u}{\partial x^2}
    = \frac{\partial^2u}{\partial r^2}\left(\frac{\partial r}{\partial x}\right)^2
      + \frac{\partial u}{\partial r}\frac{\partial^2r}{\partial x^2}
      + \frac{\partial^2u}{\partial\theta^2}\left(\frac{\partial\theta}{\partial x}\right)^2
      + \frac{\partial u}{\partial\theta}\frac{\partial^2\theta}{\partial x^2}
      + \frac{\partial^2u}{\partial\varphi^2}\left(\frac{\partial\varphi}{\partial x}\right)^2
      + \frac{\partial u}{\partial\varphi}\frac{\partial^2\varphi}{\partial x^2}.\]
  由逆变换公式求得 (所有求导项并未在下面完全列出, 因为很多项的形式是一样的)
  \[\frac{\partial r}{\partial x}
    = \frac{x}{r},\quad
    \frac{\partial^2r}{\partial x^2} = \frac{r^2-x^2}{r^3}.\]
  \[\frac{\partial\theta}{\partial x}
    = \frac{zx}{r^2\sqrt{x^2+y^2}},\quad
    \frac{\partial\theta}{\partial z}
    = \frac{-\sqrt{x^2+y^2}}{r^2}.\]
  \[\frac{\partial^2\theta}{\partial x^2}
    = \frac{zr^2y^2-2zx^2(x^2+y^2)}{r^4(x^2+y^2)^{3/2}},\quad
    \frac{\partial^2\theta}{\partial z^2}=\frac{2z\sqrt{x^2+y^2}}{r^4}\]
  \[\frac{\partial\varphi}{\partial x} = \frac{-y}{x^2+y^2},\quad
    \frac{\partial\varphi}{\partial y} = \frac{x}{x^2+y^2},\quad
    \frac{\partial\varphi}{\partial z}=0.\]
  \[\frac{\partial^2\varphi}{\partial x^2} = \frac{2xy}{(x^2+y^2)^2},\quad
    \frac{\partial^2\varphi}{\partial y^2}=\frac{-2xy}{(x^2+y^2)^2}.\]
  故
  \[\begin{split}
  \Delta u
    & = \frac{\partial^2u}{\partial r^2}\left[\left(\frac{\partial r}{\partial x}\right)^2
        + \left(\frac{\partial r}{\partial y}\right)^2
        + \left(\frac{\partial r}{\partial z}\right)^2\right]
        + \frac{\partial u}{\partial r}\left(\frac{\partial^2r}{\partial x^2}
        + \frac{\partial^2r}{\partial y^2}
        + \frac{\partial^2r}{\partial z^2}\right) \\
    & + \frac{\partial^2u}{\partial\theta^2}
        \left[\left(\frac{\partial\theta}{\partial x}\right)^2
        + \left(\frac{\partial\theta}{\partial y}\right)^2
        + \left(\frac{\partial\theta}{\partial z}\right)^2\right]
        + \frac{\partial u}{\partial\theta}\left(\frac{\partial^2\theta}{\partial x^2}
        + \frac{\partial^2\theta}{\partial y^2}+\frac{\partial^2\theta}{\partial z^2}\right) \\
    & + \frac{\partial^2u}{\partial\varphi^2}
        \left[\left(\frac{\partial\varphi}{\partial x}\right)^2
        + \left(\frac{\partial\varphi}{\partial y}\right)^2
        + \left(\frac{\partial\varphi}{\partial z}\right)^2\right]
        + \frac{\partial u}{\partial\varphi}\left(\frac{\partial^2\varphi}{\partial x^2}
        + \frac{\partial^2\varphi}{\partial y^2}
        + \frac{\partial^2\varphi}{\partial z^2}\right) \\
    & = \frac{\partial^2u}{\partial r^2}+\frac{\partial u}{\partial r}\frac{2}{r}
        + \frac{\partial^2u}{\partial\theta^2}\frac{1}{r^2}
        + \frac{\partial u}{\partial\theta}\frac{z}{r^2\sqrt{x^2+y^2}}
        + \frac{\partial^2u}{\partial\varphi^2}\frac{1}{x^2+y^2} \\
    & = \frac{1}{r^2}\frac{\partial}{\partial r}\left(r^2\frac{\partial u}{\partial r}\right)
        + \frac{1}{r^2\sin\theta}\frac{\partial}{\partial\theta}
        \left(\sin\theta\frac{\partial u}{\partial\theta}\right)
        + \frac{1}{r^2\sin^2\theta}\frac{\partial^2u}{\partial\varphi^2}.
  \end{split}\]
  证毕.
\end{proof}



\begin{exercise}
  证明: 拉普拉斯算子在柱坐标$(r,\theta,z)$下可以写成
  \[\Delta u
    = \frac{1}{r}\frac{\partial}{\partial r}\left(r\frac{\partial u}{\partial r}\right)
    + \frac{1}{r^2}\frac{\partial^2u}{\partial\theta^2}+\frac{\partial^2u}{\partial z^2}.\]
\end{exercise}

\begin{proof}
  柱坐标变换为
  \[\begin{cases}
    x = r\cos\theta, \\
    y = r\sin\theta, \\
    z = z.
  \end{cases}\]
  故
  \[\frac{\partial u}{\partial r}
    = \frac{\partial u}{\partial x}\cos\theta
      + \frac{\partial u}{\partial y}\sin\theta,\quad
    \frac{\partial u}{\partial\theta}
    = -\frac{\partial u}{\partial x}r\sin\theta
      + \frac{\partial u}{\partial y}r\cos\theta.\]
  \[\begin{split}
  \frac{\partial^2u}{\partial r^2}
    ={} & \left[\frac{\partial}{\partial x}\left(\frac{\partial u}{\partial x}\right)\cos\theta
          + \frac{\partial}{\partial y}\left(\frac{\partial u}{\partial x}\right)\sin\theta\right]\cos\theta \\
        & + \left[\frac{\partial}{\partial x}\left(\frac{\partial u}{\partial y}\right)\cos\theta
          + \frac{\partial}{\partial y}\left(\frac{\partial u}{\partial y}\right)\sin\theta\right]\sin\theta \\
    ={} & \frac{\partial^2u}{\partial x^2}\cos^2\theta
          + 2\frac{\partial^2u}{\partial x\partial y}\sin\theta\cos\theta
          + \frac{\partial^2u}{\partial y^2}\sin^2\theta.
  \end{split}\]
  \[\begin{split}
  \frac{\partial^2u}{\partial\theta^2}=
    & -\frac{\partial u}{\partial x}r\cos\theta-\frac{\partial u}{\partial y}r\sin\theta
      - \left[\frac{\partial}{\partial x}\left(\frac{\partial u}{\partial x}\right)(-r\sin\theta)
      + \frac{\partial}{\partial y}\left(\frac{\partial u}{\partial x}\right)r\cos\theta\right]r\sin\theta \\
    & + \left[\frac{\partial}{\partial x}\left(\frac{\partial u}{\partial y}\right)(-r\sin\theta)
      + \frac{\partial}{\partial y}\left(\frac{\partial u}{\partial y}\right)r\cos\theta\right]r\cos\theta \\
    ={} & -\frac{\partial u}{\partial x}r\cos\theta-\frac{\partial u}{\partial y}r\sin\theta
      + \frac{\partial^2u}{\partial\theta^2}r^2\sin^2\theta-2\frac{\partial^2u}{\partial x\partial y}r^2\sin\theta\cos\theta+\frac{\partial^2u}{\partial y^2}r^2\cos^2\theta.
  \end{split}\]
  故
  \begin{align*}
    \frac{1}{r}\frac{\partial}{\partial r}\left(r\frac{\partial u}{\partial r}\right)
      + \frac{1}{r^2}\frac{\partial^2u}{\partial\theta^2}
      + \frac{\partial^2u}{\partial z^2}
    & = \frac{1}{r}\frac{\partial u}{\partial r}+\frac{\partial^2u}{\partial r^2}
      + \frac{1}{r^2}\frac{\partial^2u}{\partial\theta^2} + \frac{\partial^2u}{\partial z^2} \\
    & = \frac{\partial^2u}{\partial x^2}+\frac{\partial^2u}{\partial y^2}
      + \frac{\partial^2u}{\partial z^2}=\Delta u. \qedhere
  \end{align*}
\end{proof}



\begin{exercise}
  证明下列函数都是调和函数:
  \begin{enumerate}[(1)]
    \item $ax+by+c$ ($a,b,c$ 为常数);
    \item $x^2-y^2$和$2xy$;
    \item $x^3-3xy^2$和$3x^2y-y^2$;
  \end{enumerate}
\end{exercise}

\begin{proof}
  直接验证即可.
\end{proof}



\begin{exercise}
  证明用极坐标表示的下列函数都满足调和方程:
  \begin{enumerate}[(1)]
    \item $\ln r$ 和 $\theta$;
    \item $r^n\cos n\theta$ 和 $r^n\sin n\theta$ ($n$为常数);
    \item $r\ln r\cos\theta-r\theta\sin\theta$ 和 $r\ln r\sin\theta+r\theta\cos\theta$.
  \end{enumerate}
\end{exercise}

\begin{proof}
  极坐标下的 Laplace 算子为
  \[\Delta u = \frac{1}{r}\frac{\partial}{\partial r}\left(r\frac{\partial u}{\partial r}\right)
    + \frac{1}{r^2}\frac{\partial^2u}{\partial\theta^2}.\]
  代入验证 $\Delta u=0$ 即可.
\end{proof}


\begin{exercise}
  用分离变量法求解由下述调和方程的第一边值问题所描述的矩阵平板 ($0\leq x\leq a,0\leq y\leq b$)
  上的稳定温度分布:
  \[\begin{cases}
    \frac{\partial^2u}{\partial x^2}+\frac{\partial^2u}{\partial y^2}=0,\\
    u(0,y)=u(a,y)=0,\\
    u(x,0)=\sin\frac{\pi x}{a},u(x,b)=0.
  \end{cases}\]
\end{exercise}

\begin{solve}
  令 $u(x,y)=X(x)Y(y)$, 代入 $\Delta u=0$ 得
  \[\frac{X''(x)}{X(x)} = -\frac{Y''(y)}{Y(y)} = \lambda.\]
  由于 $u(x,0) = X(x)Y(0) = \sin\frac{\pi x}{a}$, 故 $X(x) = C\sin\frac{\pi x}{a}$ 且求导得
  \[\frac{X''(x)}{X(x)} = \lambda = -\biggl(\frac{\pi}{a}\biggr)^2.\]
  所以
  \[Y''(y) - \biggl(\frac{\pi}{a}\biggr)^2 Y(y) = 0.\]
  解得
  \[Y(y) = C_1 \e^{\frac{\pi}{a}y} + C_2 \e^{-\frac{\pi}{a}y},\]
  结合边界条件 $Y(b) = 0$ 得
  \[Y(y) = C_3 \Bigl(\e^{\frac{(y-b)\pi}{a}} - \e^{\frac{(b-y)\pi}{a}}\Bigr)
    = 2C_3 \sinh\frac{(y-b)\pi}{a}.\]
  于是
  \[u(x,y) = X(x)Y(y) = C_4 \sinh\frac{(y-b)\pi}{a} \sin\frac{\pi x}{a}.\]
  结合 $u(x,0) = \sin\frac{\pi x}{a}$, 得
  \[u(x,y) = \frac{\sinh \frac{(y-b)\pi}{a}}{\sin h \frac{-b\pi}{a}} \sin\frac{\pi x}{a}.\qedhere\]
\end{solve}


\begin{exercise}
  在膜型扁壳渠道闸门的设计中, 为了考察闸门在水压力作用下的受力情况, 要在矩形区域
  $0\leq x\leq a$, $0\leq y\leq b$ 上求解如下的非齐次调和方程的边值问题:
  \[\begin{cases}
    \Delta u = py+q\quad (p<0, q>0\text{\ 常数}), \\
    \displaystyle\frac{\partial u}{\partial x}\biggm|_{x=0} = 0,\ u|_{x=a} = 0, \\
    u|_{y=0,y=b} = 0.
  \end{cases}\]
  试求解之.
\end{exercise}

\begin{solution}
  令 $v = u + (x^2-a^2)(fy+g)$, 通过选取 $f = -p/2$, $g = -q/2$, 则 $v$ 满足方程
  \[\begin{cases}
    \Delta v = 0, \\
    v_x|_{x=0} = v|_{x=a} = 0, \\
    v|_{y=0} = -\frac{q}{2}(x^2-a^2) = \alpha(x), \\
    v|_{y=b} = -\frac{1}{2}(bp+q)(x^2-a^2) = \beta(x).
  \end{cases}\]
  再利用分离变量法求解即可.
\end{solution}


\begin{exercise}
  举例说明在二维 Laplace 方程的 Dirichlet 外问题中, 如对解 $u(x,y)$ 不加在无穷远处
  为有界的限制, 那么定解问题的解不是唯一的.
\end{exercise}

\begin{proof}
  考虑区域 $\varOmega = \{(x,y)\mid x^2+y^2 > 1\}$ 以及相应的 Dirichlet 外问题
  \[\Delta u = 0\text{ in }\varOmega,\quad u = 1\text{ on }\partial\varOmega.\]
  显然 $u\equiv 1$ 和 $u = c\ln\frac{1}{r}+1$ 都为对应的解.
\end{proof}


\begin{exercise}
  设
  \[J(u) = \iiint_{\varOmega} \frac12 \biggl[\biggl(\frac{\partial u}{\partial x}^2\biggr)
    + \biggl(\frac{\partial u}{\partial y}^2\biggr)
    + \biggl(\frac{\partial u}{\partial z}^2\biggr)\biggr] \diff x\diff y\diff z
    + \iint_{\varGamma} \biggl(\frac12 \sigma u^2 - gu\biggr) \diff s,\]
  变分问题的提法为: 求 $u\in V$, 使
  \[J(u) = \min_{v\in V} J(v),\]
  其中 $V = C^2(\varOmega)\cap C^1(\closure{\varOmega})$.
  试导出与此变分问题等价的边值问题, 并证明它们的等价性.
\end{exercise}

\begin{proof}
  等价的边值问题为
  \[\begin{cases}
    \Delta u = 0, \\
    \Bigl(\frac{\partial u}{\partial \bm{n}} + \sigma u\Bigr)\Bigm|_{\varGamma} = g.
  \end{cases}\]
\end{proof}



\section{格林公式及其应用}

\begin{exercise}
  证明 (2.7) 式对于 $M_0$ 在 $\varOmega$ 外与 $\Gamma$ 上的情形成立.
\end{exercise}

\begin{proof}
  当 $M_0$ 在 $\varOmega$ 外时, $v=\frac{1}{r_{M_0M}}$ 在区域 $\varOmega$ 内无奇异点, 故由格林第二公式得
  \[\iiint_{\varOmega} \left(u\Delta\frac{1}{r}-\frac{1}{r}\Delta u\right)\diff V
    = \iint_{\Gamma}\left(u\frac{\partial}{\partial\vec{n}}\left(\frac{1}{r}\right)
      - \frac{1}{r}\frac{\partial u}{\partial\vec{n}}\right) \diff S.\]
  在 $\varOmega$ 内 $\Delta u=0$, $\Delta\frac{1}{r}=0$, 故
  \[-\iint_{\Gamma}\left(u\frac{\partial}{\partial\vec{n}}\left(\frac{1}{r}\right)
    - \frac{1}{r}\frac{\partial u}{\partial\vec{n}}\right)\diff S = 0.\]

  当 $M_0$ 在 $\Gamma$ 上时, 将以 $M_0$ 为球心, 
  以充分小正数 $\varepsilon$ 为半径的球与 $\varOmega$ 相交的部分记为 $K_{\varepsilon}$,
  将 $K_{\varepsilon}$ 的包含于 $\varOmega$ 内的边界记为 $\Gamma_{\varepsilon}$,
  且记 $\partial(\varOmega\backslash K_{\varepsilon})-\Gamma_{\varepsilon}=\Gamma_{\varepsilon}'$,
  则由格林第二公式得
  \[0 = \iiint_{\varOmega\backslash K_{\varepsilon}}
    \left(u\Delta\frac{1}{r} - \frac{1}{r}\Delta u\right)\diff V
    = \iint_{\Gamma_{\varepsilon}\bigcup\Gamma_{\varepsilon}'} \left(u\frac{\partial}{\partial\vec{n}}
      \left(\frac{1}{r}\right)-\frac{1}{r}\frac{\partial u}{\partial\vec{n}}\right) \diff S.\]
  故
  \[-\iint_{\Gamma_{\varepsilon}'} \left(u\frac{\partial}{\partial\vec{n}}\left(\frac{1}{r}\right)
    -\frac{1}{r}\frac{\partial u}{\partial\vec{n}}\right) \diff S
    = \iint_{\Gamma_{\varepsilon}}\left(u\frac{\partial}{\partial\vec{n}}
      \left(\frac{1}{r}\right) - \frac{1}{r}\frac{\partial u}{\partial\vec{n}}\right)\diff S.\]
  而
  \[\iint_{\Gamma_{\varepsilon}} \left(u\frac{\partial}{\partial\vec{n}}\left(\frac{1}{r}\right)
    -\frac{1}{r}\frac{\partial u}{\partial\vec{n}}\right) \diff S
    = \iint_{\Gamma_{\varepsilon}}\left(\frac{u}{\varepsilon^2}
      - \frac{1}{\varepsilon}\frac{\partial u}{\partial\vec{n}}\right) \diff S.\]
  令 $\varepsilon\to 0$, 注意到 $\Gamma$ 充分光滑
  (这意味着面积 $S(\Gamma_{\varepsilon})\to 2\pi\varepsilon^2$), 所以有
  \[\lim_{\varepsilon\to 0} \iint_{\Gamma_{\varepsilon}}
    \left(\frac{u}{\varepsilon^2}-\frac{1}{\varepsilon}\frac{\partial u}{\partial\vec{n}}\right)\diff S
    = 2\pi u(M_0).\]
  同时 $\lim_{\varepsilon\to0}\Gamma_{\varepsilon}'=\Gamma$, 因此最终得到
  \[-\iint_{\Gamma} \left(u\frac{\partial}{\partial\vec{n}}\left(\frac{1}{r}\right)-\frac{1}{r}
    \frac{\partial u}{\partial\vec{n}}\right)\diff S = 2\pi u(M_0). \qedhere\]
\end{proof}


\begin{exercise}
  若函数 $u(x,y)$ 是单位圆周上的调和函数, 又它在单位圆周上的数值已知为 $u=\sin\theta$,
  其中 $\theta$ 表示极角, 问函数 $u$ 在原点之值等于多少?
\end{exercise}

\begin{proof}
  由平均值公式知原点之值为
  \[u(O) = \frac{1}{2\pi}\int_{\Gamma}\sin\theta\diff s
    = \frac{1}{2\pi}\int_0^{2\pi}\sin\theta\diff\theta = 0.\qedhere\]
\end{proof}


\begin{exercise}[4]
  证明: 当 $u(M)$ 在闭曲面 $\Gamma$ 的外部调和, 并且在无穷远处成立
  \[u(M) = O\biggl(\frac{1}{r_{OM}}\biggr),\quad
    \frac{\partial u}{\partial r} = O\biggl(\frac{1}{r_{OM}^2}\biggr)\quad
    (r_{OM}\to\infty),\]
  而 $M_0$ 是 $\Gamma$ 外任意一点, 则公式 (2.6) 仍成立.
\end{exercise}

\begin{proof}
  取以 $M_0$ 为球心, 以 $R$ (充分大)为半径的球 $K_R$ 使其包含曲面 $\Gamma$,
  并记该球去掉闭曲面 $\Gamma$ 内部区域后得到的部分为 $\varOmega_R$. 将 $K_R$ 的边界记为 $\Gamma_R$,
  再取以 $M_0$ 为球心, 以 $\epsilon$ 为半径的球 $K_{\epsilon}$ 使其完全包含在区域 $\varOmega_R$中,
  将 $K_{\epsilon}$ 的边界记为 $\Gamma_{\epsilon}$.
  取 $r=r_{MM_0}$, 则由格林第二公式得
  \begin{equation}
    \begin{aligned}
    0 & = \iiint_{\varOmega_R\backslash K_{\epsilon}}\left(u\Delta\frac{1}{r}
        - \frac{1}{r}\Delta u\right)\diff V \\
      & = \iint_{\Gamma\bigcup\Gamma_R\bigcup\Gamma_{\epsilon}}\left(u\frac{\partial}{\partial\vec{n}}\left(\frac{1}{r}\right)-\frac{1}{r}\frac{\partial u}{\partial\vec{n}}\right)\diff S.
    \end{aligned} \tag{$\star$}
  \end{equation}
  因为 $u(M)=O(\frac{1}{r})$, $\frac{\partial u}{\partial r}=O(\frac{1}{r^2})$ $(r\to\infty)$,
  所以当 $R\to +\infty$ 时,
  \[\iint_{\Gamma_R}\left(u\frac{\partial}{\partial\vec{n}}\left(\frac{1}{r}\right)
    - \frac{1}{r}\frac{\partial u}{\partial\vec{n}}\right)\diff S
    = \iint_{\Gamma_R}\left(\frac{-u}{R^2}-\frac{1}{R}\frac{\partial u}{\partial r}\right)\diff S = 0.\]
  又因为当 $\epsilon\to 0$ 时,
  \[\iint_{\Gamma_{\epsilon}}\left(u\frac{\partial}{\partial\vec{n}}\left(\frac{1}{r}\right)
    - \frac{1}{r}\frac{\partial u}{\partial\vec{n}}\right)\diff S
    = \iint_{\Gamma_{\epsilon}}\left(\frac{u}{\epsilon^2}
    + \frac{1}{\epsilon}\frac{\partial u}{\partial r}\right)\diff S
    \to 4\pi u(M_0).\]
  在 ($\star$) 式中令 $R\to +\infty$, $\epsilon\to 0$, 即得
  \[u(M_0)
  =-\frac{1}{4\pi}\iint_{\Gamma}\left(u\frac{\partial}{\partial\vec{n}}\left(\frac{1}{r}\right)
  -\frac{1}{r}\frac{\partial u}{\partial\vec{n}}\right)\diff S. \qedhere\]
\end{proof}


\begin{exercise}
  证明调和方程 Dirichlet 外问题解的稳定性.
\end{exercise}

\begin{proof}
  在闭曲面 $\Gamma$ 上给定两个函数 $f,f^*$, 并且在$\Gamma$上满足$|f-f^*|\leq\epsilon$,
  设 $u,u^*$ 是相应的狄利克雷外问题的解, 以 $\Gamma_R$ 表示半径为 $R$ 的球面, 令$v=u-u^*$, 因为
  \[\lim_{r\to\infty}v(x,y,z)=0.\]
  所以存在 $R_0$, 使得在 $\Gamma_{R_0}$ 及其外部满足 $|v|\leq\epsilon$,
  在 $\Gamma$ 和 $\Gamma_{R_0}$ 围成的有界区域中,
  利用极值原理知 $|v|\leq\epsilon$, 故在 $\Gamma$ 的外部满足 $|v|\leq\epsilon$,
  由此证明了狄利克雷外问题的解是稳定的.
\end{proof}


\begin{exercise}
  对于二阶偏微分方程
  \[\sum_{i,j=1}^n a_{ij} u_{x_ix_j} + \sum_{i=1}^n b_i u_{x_i} + cu = 0,\]
  其中 $a_{ij}$, $b_i$, $c$ ($i,j=1,\dots,n$) 均为常数. 假设存在常数 $\lambda>0$, 使得
  \[\sum_{i,j=1}^n a_{ij}\xi_i\xi_j \geq \lambda |\xi|^2,\quad \forall\xi\in\mathbb{R}^n.\]
  又设 $c>0$, 证明极值原理: 若 $u$ 在 $\varOmega$ 中满足方程, 在 $\varOmega\cup\Gamma$ 上连续,
  则 $u$ 不能在 $\varOmega$ 的内部达到正的最大值或负的最小值.
\end{exercise}

\begin{proof}
  假设 $u$ 在点 $x_0\in\varOmega$ 处达到正的最大值, 则
  \[\nabla u(x_0) = 0,\qquad (D_{ij} u(x_0))\leq 0.\]
  因此
  \[\sum_{i,j=1}^n a_{ij} u_{x_ix_j}(x_0) \leq 0,
    \quad \sum_{i=1}^n b_i u_{x_i}(x_0) = 0,\quad cu(x_0)<0.\]
  从而
  \[\sum_{i,j=1}^n a_{ij} u_{x_ix_j}(x_0) + \sum_{i=1}^n b_i u_{x_i}(x_0) + cu(x_0) <0,\]
  与条件矛盾.
\end{proof}


\section{格林函数}

\[u(M_0) = \iint_{\Gamma}\left[\frac{1}{4\pi r_{M_0M}}\frac{\partial u}{\partial\vec{n}}-u\frac{\partial}{\partial\vec{n}}\frac{1}{4\pi r_{M_0M}}\right]\diff S_M\]
\[\iint_{\Gamma}\left(g\frac{\partial u}{\partial\vec{n}}-u\frac{\partial g}{\partial\vec{n}}\right)\diff S_M=0\]
相减得
\[u(M_0)=\iint_{\Gamma}\left(G\frac{\partial u}{\partial\vec{n}}-u\frac{\partial G}{\partial\vec{n}}\right)\diff S_M,\mbox{其中}G(M,M_0)=\frac{1}{4\pi r_{M_0M}}-g(M,M_0)\]

\begin{note}
  格林函数的五点性质的证明.

  \begin{property}
    格林函数 $G(M, M_0)$ 除 $M=M_0$ 一点外处处满足方程 (1.1), 而当 $M\to M_0$ 时,
    $G(M, M_0)$ 趋于无穷大, 其阶数和 $\frac{1}{4\pi r_{M_0M}}$ 相同.
  \end{property}

  \begin{proof}
    除了点$M=M_0$外,$\frac{1}{4\pi r_{M_0M}}$调和,又因为$g(M,M_0)$在$\varOmega$内调和,故$G(M,M_0)$除了$M=M_0$外处处调和,由极值原理知$g(M,M_0)$在$\varOmega$上有界,故
    \[\lim_{M\to M_0}G(M,M_0)=\lim_{M\to M_0}\left(\frac{1}{4\pi r_{M_0M}}-g(M,M_0)\right)=\infty\mbox{且和}\frac{1}{4\pi r_{M_0M}}\mbox{同阶}\]
  \end{proof}

  \begin{property}
    在边界上格林函数 $G(M, M_0)$ 恒等于零.
  \end{property}

  \begin{proof}
    由$g(M,M_0)$的定义知$G(M,M_0)|_{\Gamma}=0$.
  \end{proof}

  \begin{property}
    在区域 $\varOmega$ 上成立着不等式:
    \[0 < G(M, M_0) < \frac{1}{4\pi r_{M_0M}}.\]
  \end{property}
  
  \begin{proof}
    \[0<G(M,M_0)<\frac{1}{4\pi r_{M_0M}}\Leftrightarrow 0<g(M,M_0)<\frac{1}{4\pi r_{M_0M}}\]
    由极值原理知$g(M,M_0)>0$是显然的,下面证明$g(M,M_0)<\frac{1}{4\pi r_{M_0M}}$:

    取$\delta$足够小使得$\frac{1}{4\pi r_{M_0M}}>g(M,M_0)$在$B(M_0,\delta)$上成立(这样的$\delta$显然是可以取到的),记$D=\varOmega\backslash\overline{B(M_0,\delta)}$,则$\frac{1}{4\pi r_{M_0M}}-g(M,M_0)$在$D$上调和,且
    \[\min_{\partial D}\left(\frac{1}{4\pi r_{M_0M}}-g(M,M_0)\right)=0\]
    故由极值原理知在$D$上成立$\frac{1}{4\pi r_{M_0M}}>g(M,M_0)$,从而在$\varOmega$上成立$\frac{1}{4\pi r_{M_0M}}>g(M,M_0)$.
  \end{proof}

  \begin{property}
    格林函数 $G(M, M_0)$ 在自变量 $M$ 及参变量 $M_0$ 之间具有对称性, 即设
    $M_1$, $M_2$ 为区域中的亮两点, 则
    \[G(M_1, M_2) = G(M_2, M_1).\]
  \end{property}
  
  \begin{proof}
    记$D_{\epsilon}=\varOmega\backslash(B(M_1,\epsilon)\bigcup B(M_2,\epsilon))$,
    再记$w(M)=G(M,M_2),v(M)=G(M,M_1)$,则我们要证明$w(M_1)=v(M_2)$,
    由定义知$w(M)$和$v(M)$都是$D_{\epsilon}$上的调和函数,故由格林公式得
    \[\iint_{\partial D_{\epsilon}}\left(w\frac{\partial v}{\partial\vec{n}}-v\frac{\partial w}{\partial\vec{n}}\right)\diff S=0\]
    故
    \[\iint_{\partial B(M_1,\epsilon)}\left(w\frac{\partial v}{\partial\vec{n}}-v\frac{\partial w}{\partial\vec{n}}\right)\diff S=\iint_{\partial B(M_2,\epsilon)}\left(v\frac{\partial w}{\partial\vec{n}}-w\frac{\partial v}{\partial\vec{n}}\right)\diff S\]
    当$\epsilon\to0+$时,在$\partial B(M_1,\epsilon)$上,$v=O(\frac{1}{\epsilon}),\frac{\partial v}{\partial\vec{n}}=O(\frac{1}{\epsilon^2}),\frac{\partial v}{\partial\vec{n}}=-\frac{1}{4\pi\epsilon^2}+\mbox{有界量}$,故
    \[\iint_{\partial B(M_1,\epsilon)}w\frac{\partial v}{\partial\vec{n}}\diff S=\iint_{\partial B(M_1,\epsilon)}-\frac{1}{4\pi\epsilon^2}w\diff S+\iint_{\partial B(M_1,\epsilon)}\mbox{有界量}\cdot w\diff S\to-w(M_1)\]取极限$\epsilon\to0+$,即得$w(M_1)=v(M_2)$.
  \end{proof}

  \begin{property}
    $\iint_{\varGamma} \frac{\partial G(M, M_0)}{\partial n} \diff S_M = -1$.
  \end{property}
  
  \begin{proof}
    设 $\Gamma_{\epsilon}$ 是以 $M_0$为球心,
    以 $\epsilon$ 为半径的球面, 并且其包含在 $\varOmega$ 当中, 则
    \[\begin{split}
      \iint_{\Gamma}\frac{\partial G(M,M_0)}{\partial\vec{n}}\diff S_M
      & = \iint_{\Gamma}\frac{\partial}{\partial\vec{n}}\left(\frac{1}{4\pi r_{M_0M}}\right)\diff S_M-\iint_{\Gamma}\frac{\partial g(M,M_0)}{\partial\vec{n}}\diff S_M \\
      & = \iint_{\Gamma}\frac{\partial}{\partial\vec{n}}
          \left(\frac{1}{4\pi r_{M_0M}}\right)\diff S_M\quad (\text{Theorem 2.1}) \\
      & = \iint_{\Gamma_{\epsilon}}\frac{\partial}{\partial\vec{n}}
          \left(\frac{1}{4\pi r_{M_0M}}\right)\diff S_M\quad (\text{Theorem 2.1}) \\
      & = \iint_{\Gamma_{\epsilon}}\frac{-1}{4\pi\epsilon^2}\diff S_M = -1.
    \end{split}\]
    另法:考虑定解问题(老师上课时解法)
    \[\begin{cases}
    \Delta u=0(in\;\varOmega)\\u|_{\Gamma=1}
    \end{cases}\]
    由极值原理知解为$u\equiv1$,代入(3.4)式即得结论
  \end{proof}

  二维情形圆的格林函数取为$G(M,M_0)=\frac{1}{2\pi}\left(\ln\frac{1}{r_{M_0M}}-\ln\frac{R}{\rho_0}\frac{1}{r_{M_1M}}\right)$也是为了使得$\int_{\Gamma}\frac{\partial G(M,M_0)}{\partial\vec{n}}\diff s=-1$成立,证明思路同上.
\end{note}



\begin{exercise}[5]
  求半圆区域上狄利克雷问题的格林函数.
\end{exercise}

\begin{solve}
  设上半圆区域为 $D$, 半径为$R$, $A(x_0,y_0)\in D,M(x,y)\in D$,
  点 $A$ 关于边界圆的反演点为 $A_1(R^2\frac{x_0}{\sqrt{x_0^2+y_0^2}}$,
  $R^2\frac{y_0}{\sqrt{x_0^2+y_0^2}})$,
  记 $A$ 与 $A_1$ 关于 $x$ 轴的对称点分别为 $A_2$, $A_3$,
  再记 $MA=r$, $MA_1=r_1$, $MA_2=r_2$, $MA_3=r_3$, 则格林函数为
  \[G(M,M_0)=\frac{1}{2\pi}\ln\frac{r_1r_2}{rr_3}.\qedhere\]
\end{solve}


\begin{exercise}[7]
  证明二维调和函数的可去奇点定理:
  若 $A$ 是调和函数 $u(M)$ 的孤立奇点, 在 $A$ 点邻域中成立着
  \[u(M) = o\biggl(\ln\frac{1}{r_{AM}}\biggr),\]
  则此时可以重新定义 $u(M)$ 在 $M=A$ 的值, 使它在 $A$ 点也是调和的.
\end{exercise}

\begin{proof}
  设 $K$ 是以 $A$ 为圆心, 以$R$为半径的圆,并且其完全包含在$A$的那个所考察的邻域中,
  以 $u$ 在 $K$ 上的值为边界条件, 在 $K$ 内求拉普拉斯方程的解,记为 $v$,
  下面证明在整个 $K$ 内除了点 $A$ 外 $u=v$, 令 $w=u-v$, 则
  \[\lim_{M\to A}\frac{w(M)}{\ln\frac{1}{r_{AM}}}=0.\]
  此外在圆 $K$ 的边界上 $w=0$. 作函数
  \[w_{\epsilon}(M) = \epsilon\left(\ln\frac{1}{r_{AM}}-\ln\frac{1}{R}\right).\]
  具有如下的性质: (1)在$K$的边界上$w_{\epsilon}(M)=0$;在$K$内$w_{\epsilon}(M)>0$.
  (2) 在$r=\delta$和$r=R$所包围的同心圆环上是调和函数,这里$\delta$是任意小的正数.
  因为
  \[\lim_{M\to A}\frac{w_{\epsilon}(M)}{\ln\frac{1}{r_{AM}}}=\epsilon,\]
  所以存在 $\delta'$, 使得当 $r_{AM}<\delta'$时
  \[\frac{w_{\epsilon}(M)}{\ln\frac{1}{r_{AM}}}>\frac{\epsilon}{2}.\]
  又因为
  \[\lim_{M\to A}\frac{w(M)}{\ln\frac{1}{r_{AM}}}=0,\]
  所以存在 $\delta''$,使得当$r_{AM}<\delta''$时
  \[\left|\frac{w(M)}{\ln\frac{1}{r_{AM}}}\right|<\frac{\epsilon}{2}\]
  令$\delta=\min(\delta',\delta'')$,则在圆$r_{AM}=\delta$上有
  \[\left|\frac{w(M)}{\ln\frac{1}{r_{AM}}}\right|
    \leq\frac{w_{\epsilon}(M)}{\ln\frac{1}{r_{AM}}}\Rightarrow|w(M)|\leq w_{\epsilon}(M).\]
  故由极值原理推论2知对于 $r=\delta$ 和 $r=R$ 所包围的同心圆环中的任意一点 $M$, 成立
  \[|w(M)|\leq w_{\epsilon}(M).\]
  令 $\epsilon\to0$, 即得 $w(M)=0,\forall M\in K\backslash A$.
\end{proof}


\begin{exercise}
  证明: 如果三维调和函数 $u(M)$ 在奇点 $A$ 附近表示成 $\frac{N}{r_{AM}^{\alpha}}$,
  其中常数 $0<\alpha\leq 1$, 而 $N$ 是不为零的光滑函数, 则当 $M\to A$ 时它趋于
  无穷大的阶数必与 $\frac{1}{r_{AM}}$ 同阶, 即 $\alpha=1$.
\end{exercise}

\begin{proof}
  假设$\alpha<1$,则
  \[\lim_{M\to A}r_{AM}u(M) = \lim_{M\to A}Nr_{AM}^{1-\alpha} = 0.\]
  由可去奇点定理知 $A$ 是可去奇点, 矛盾, 故 $\alpha=1$.
\end{proof}


\begin{exercise}
  Try to find a function $u$ such that it is harmonic in a circle of radius $\alpha$
  and takes the following values on the boundary $C$:
  \begin{enumerate}[(1)]
    \item $u|_C = A\cos\varphi$;
    \item $u|_C = A + B\sin\varphi$.
  \end{enumerate}
\end{exercise}

9.解:利用泊松公式(3.13)式
\[u(\rho_0,\varphi_0)=\frac{1}{2\pi}\int_0^{2\pi}\frac{(R^2-\rho_0^2)f(\varphi)}{R^2+\rho_0^2-2R\rho_0\cos(\varphi-\varphi_0)}\diff\varphi\]
(1)\[\begin{split}
u(\rho_0,\varphi_0)&=\frac{1}{2\pi}\int_0^{2\pi}\frac{(R^2-\rho_0^2)A\cos(\varphi)}{R^2+\rho_0^2-2R\rho_0\cos(\varphi-\varphi_0)}\diff\varphi\\
&=\frac{A(R^2-\rho_0^2)}{2\pi}\int_0^{2\pi}\frac{\cos(\varphi)}{R^2+\rho_0^2-2R\rho_0\cos(\varphi-\varphi_0)}\diff\varphi\\
&=\frac{A(R^2-\rho_0^2)}{2\pi}\int_{0}^{2\pi}\frac{\cos(\theta+\varphi_0)}{R^2+\rho_0^2-2R\rho_0\cos\theta}\diff\theta(\mbox{换元}\theta=\varphi-\varphi_0)\\
&=\frac{A(R^2-\rho_0^2)}{2\pi}\int_{-\pi}^{\pi}\frac{\cos\theta\cos\varphi_0-\sin\theta\sin\varphi_0}{R^2+\rho_0^2-2R\rho_0\cos\theta}\diff\theta\\
&=\frac{A(R^2-\rho_0^2)}{2\pi}\int_{-\pi}^{\pi}\frac{\cos\theta\cos\varphi_0}{R^2+\rho_0^2-2R\rho_0\cos\theta}\diff\theta\\
&=\frac{2A(R^2-\rho_0^2)\cos\varphi_0}{2\pi}\int_0^{\pi}\frac{\cos\theta}{R^2+\rho_0^2-2R\rho_0\cos\theta}\diff\theta\\
&=\frac{2A(R^2-\rho_0^2)\cos\varphi_0}{2\pi}\frac{\pi\rho_0}{R(R^2-\rho_0^2)}\\
&=\frac{A}{R}\rho_0\cos\varphi_0=\frac{A}{\alpha}\rho_0\cos\varphi_0
\end{split}\]
故\[u(\rho,\varphi)=\frac{A}{\alpha}\rho\cos\varphi\]
(2)因为$u|_C=A+B\sin\varphi=A+B\cos(\varphi-\frac{\pi}{2})$,故由叠加原理及(1)中结果知
\[u(\rho,\varphi)=A+\frac{B}{\alpha}\rho\cos\left(\varphi-\frac{\pi}{2}\right)=A+\frac{B}{\alpha}\rho\sin\varphi\]


\begin{exercise}
  Derive the solution of the Dirichlet problem of 2-dimensional Laplace
  equation in the half plane:
  \[\begin{cases}
    \Delta u = u_{xx} + u_{yy} = 0, \quad y>0, \\
    u|_{y=0} = f(x).
  \end{cases}\]
\end{exercise}

\begin{solve}
  \[G(M,M_0)=\frac{1}{2\pi}\left[\ln\frac{1}{\sqrt{(x-x_0)^2+(y-y_0)^2}}-\ln\frac{1}{\sqrt{(x-x_0)^2+(y+y_0)^2}}\right]\]
  注意到$\frac{\partial}{\partial\vec{n}}=-\frac{\partial}{\partial y}$,故
  \[\begin{split}
  u(x_0,y_0)&=-\int_{\Gamma}f(x)\frac{\partial G(M,M_0)}{\partial\vec{n}}\diff x\\
  &=\frac{1}{2\pi}\int_{-\infty}^{\infty}f(x)\frac{\partial}{\partial y}\left[\ln\frac{1}{\sqrt{(x-x_0)^2+(y-y_0)^2}}-\ln\frac{1}{\sqrt{(x-x_0)^2+(y+y_0)^2}}\right]\Bigg|_{y=0}\diff x\\
  &=\frac{1}{2\pi}\int_{-\infty}^{\infty}f(x)\left(-\frac{1}{2}\frac{2(y-y_0)}{(x-x_0)^2+(y-y_0)^2}+\frac{1}{2}\frac{2(y+y_0)}{(x-x_0)^2+(y+y_0)^2}\right)\bigg|_{y=0}\diff x\\
  &=\frac{1}{\pi}\int_{-\infty}^{\infty}f(x)\frac{y_0}{(x-x_0)^2+y_0^2}\diff x
  \end{split}\]
  因此二维调和方程在半平面上的狄利克雷问题的解为:
  \[u(x,y)=\frac{1}{\pi}\int_{-\infty}^{\infty}\frac{f(\xi)y}{(\xi-x)^2+y^2}\diff\xi. \qedhere\]
\end{solve}


\section{强极值原理,第二边值问题解的唯一性}

\begin{exercise}
  试用强极值原理来证明极值原理: 对不恒等于常数的调和函数$u(x,y,z)$,
  其在区域 $\varOmega$ 的任何内点上的值不可能达到它在 $\varOmega$ 上的上界或下界.
\end{exercise}

\begin{proof}
  假设调和函数 $u(x,y,z)$ 不恒等于常数, 且在区域 $\varOmega$ 内部某点达最小值 $m$, 记
  \[E = \{ M\in\varOmega \mid u(M) = m\}.\]
  则由 $u$ 的连续性知 $E$ 是相对闭集, 由于 $u$ 不恒为常数, 故 $\varOmega\setminus E$ 为非空开集,
  取点 $M_0\in\varOmega\setminus E$ 使得 $\dist(M_0, E) < \dist(M_0, \partial\varOmega)$.
  取以点 $M_0$ 为球心, 以 $\dist(M_0, E)$ 为半径的球 $B$, 取 $M_1\in \partial B\cap E$.
  对于 $B$ 内任一点 $M$ 均有 $u(M)>u(M_1)$, 故由强极值原理
  \[\frac{\partial u}{\partial\vec{\nu}}\bigg|_{M_1} > 0,\]
  其中 $\vec{\nu}$ 与 $B$ 在点 $M_1$ 处的内法线方向成锐角,
  但是由于 $M_1$ 是 $\varOmega$ 内部的最小值点, 故对于任意方向 $\vec{l}$ 均有
  \[\frac{\partial u}{\partial\vec{l}}\bigg|_{M_1} = 0.\]
  矛盾.
\end{proof}


\begin{exercise}
  利用极值原理和强极值原理证明: 当区域 $\varOmega$ 的边界 $\varGamma$
  满足定理 4.2 中的条件时, 调和方程第三边值问题
  \[\biggl(\frac{\partial u}{\partial \bm{n}} + \sigma u\biggr)\bigg|_{\varGamma}
    = f \quad (\sigma>0)\]
  的解的唯一性.
\end{exercise}

\begin{proof}
  只需要证明满足边界条件 $(\frac{\partial u}{\partial\bm{n}}+\sigma u)|_{\varGamma}=0$ 的只有零解即可,
  下面分两种情况讨论.

  对于第三边值问题的内问题: 假设 $u$ 不恒为常数, 则由极值原理知 $u$ 在 $\varGamma$ 上取得最大值和最小值,
  记在 $M_1$ 处取到最小值, 在 $M_2$ 处取得最大值, 则
  \[\frac{\partial u}{\partial\bm{n}}(M_1) + \sigma u(M_1) = 0
    \Rightarrow u(M_1) = -\frac{1}{\sigma}\frac{\partial u}{\partial\bm{n}}(M_1)>0.\]
  \[\frac{\partial u}{\partial\bm{n}}(M_2) + \sigma u(M_2) = 0
    \Rightarrow u(M_2) = -\frac{1}{\sigma}\frac{\partial u}{\partial\bm{n}}(M_2)<0.\]
  故 $u(M_2)<u(M_1)$, 矛盾, 故假设不成立, 所以
  \[u\equiv C\Rightarrow\frac{\partial u}{\partial\bm{n}}\bigg|_{\varGamma} = 0
    \Rightarrow u|_{\varGamma}=0\Rightarrow u\equiv 0.\]
  
  对于第三边值问题的外问题: 记边界 $\varGamma$ 的外部为 $\varOmega'$,
  假设存在 $M_0\in\varOmega'$, 使得 $u(M_0)>0$, 由于 $\lim_{M\to\infty}u(M)=0$,
  故存在充分大的 $R$, 使得在 $\varGamma_R = \{r=R\}$ 上成立 $|u|<u(M_0)$,
  则由极值原理知 $u$ 的最大值只能在 $\varGamma$ 上取, 设最大值点为 $M_1\in\varGamma$, 则
  \[\frac{\partial u}{\partial\bm{n}}\bigg|_{M_1}+\sigma u(M_1)>0,\]
  与边界条件相矛盾, 假设 $u(M_0)<0$ 同样可以导出矛盾, 故$u\equiv 0$.
\end{proof}


\begin{exercise}
  说明在证明强极值原理过程中, 不可能作出一个满足条件 (1) 和 (3) 的辅助函数 $v(x,y,z)$,
  使它在整个球 $\closure{B}_R = \{x^2+y^2+z^2 \leq R^2\}$ 内满足 $\Delta v>0$.
\end{exercise}

\begin{proof}
  若在 $B_R$ 上有 $\Delta v>0$, 则
  \[\max_{\closure{B}_R} v = \max_{\partial B_R} v = 0.\]
  又因为 $\frac{\partial v}{\partial r} < 0$, 所以
  \[\min_{\closure{B}_R} v = \min_{\partial B_R} v = 0.\]
  结合二者即得 $v\equiv 0$, 与 $\Delta v>0$ 矛盾.
\end{proof}


\begin{exercise}
  设 $\varOmega$ 为 $\mathbb{R}^3$ 中有界区域, 边界为 $\varGamma$, $u$ 为定解问题
  \[\begin{cases}
    - \Delta u + cu = f, \\
    \biggl(\frac{\partial u}{\partial \bm{n}} + \sigma u\biggr)\biggm|_{\partial\varOmega} = g
  \end{cases}\]
  的解, 其中 $c,f,g,\sigma>0$, 证明在 $\closure{\varOmega}$ 上 $u>0$.
\end{exercise}


\begin{exercise}[6]
  对于一般的椭圆型方程
  \[\sum_{i,j=1}^n a_{ij}\frac{\partial^2u}{\partial x_i\partial x_j}
    + \sum_{i=1}^n b_i \frac{\partial u}{\partial x_i} + cu = 0,\]
  其中矩阵 $(a_{ij})$ 正定, 即存在常数 $\alpha>0$ 使得
  \[\sum_{i,j=1}^n a_{ij}\xi_i\xi_j \geq \alpha|\xi|^2\quad
    \forall\xi\in \mathbb{R}^n.\]
  又设 $c\leq 0$, 试证明它的解也成立着强极值原理.
  也就是说, 如果 $u(M)$ 在球 $|x|<R$ 内满足上述方程, 在闭球 $|x|\leq R$ 上连续,
  在球面上一点 $M_0$ 处取到非正的最小值, 且在该点沿 $\bm{\nu}$ 方向的方向导数
  $\frac{\partial u}{\partial \bm{\nu}}$ 存在, 其中 $\bm{\nu}$ 与球的内法线方向成锐角,
  则在 $M_0$ 点有 $\frac{\partial u}{\partial \bm{\nu}}>0$.
\end{exercise}

\begin{proof}
  若 $u$ 在球面上一点 $M_0$ 取非正的最小值, 即 $u(M_0)\leq 0$,
  且对球内任一点 $M$ 有 $u(M)>u(M_0)$, 因此在 $M_0$ 点有
  \[\frac{\partial u}{\partial\bm{\nu}}\geq 0.\]
  现在需要证明上式中的等号不能成立, 构造函数
  \[v(x) = \e^{-a\sum_{i=1}^nx_i^2} - \e^{-aR^2} = \e^{-a|x|^2} - \e^{-aR^2},\]
  其中 $a$ 为待定的正常数, 则 $v$ 满足如下性质:
  \begin{enumerate}[(1)]
    \item 在球面 $|x|=R$ 上 $v=0$;
    \item 通过适当选取 $a$, 在区域 $D=\{ R/2 \leq |x| \leq R\}$ 内
      \[Lv = \sum_{i,j=1}^na_{ij}\frac{\partial^2v}{\partial x_i\partial x_j}
        + \sum_{i=1}^nb_i\frac{\partial v}{\partial x_i}+cv > 0.\]
      事实上,因为
      \[\frac{\partial v}{\partial x_i} = -2a x_i \e^{-a|x|^2},\]
      \[\frac{\partial^2v}{\partial x_i\partial x_j} = 4a^2 x_i x_j 
        \e^{-a|x|^2},\quad i\neq j,\]
      \[\frac{\partial^2v}{\partial x_i^2} = 4a^2 x_i^2 \e^{-a|x|^2}
        -2a \e^{-a|x|^2}.\]
      所以
      \begin{align*}
      Lv
      & = 4a^2\left(\sum_{i,j=1}^na_{ij}x_ix_j\right) \e^{-a|x|^2}
        - 2a \sum_{i=1}^n (b_ix_i + a_{ii}) \e^{-a|x|^2} \\
      & \quad + c \left(\e^{-a|x|^2} - \e^{-aR^2}\right) \\
      & = \e^{-a|x|^2} \biggl\{ 4a^2\sum_{i,j=1}^n a_{ij} x_i x_j
        - 2a\sum_{i=1}^n (b_i x_i + a_{ii}) + c \left(1 - \e^{-a(R^2-|x|^2)}\right)\biggr\}
      \end{align*}
      因为 $\sum_{i,j=1}^n a_{ij} x_i x_j \geq \alpha |x|^2 \geq \frac{\alpha R^2}{4} > 0$,
      故当 $a$ 充分大时在区域 $D$ 内 $Lv > 0$.
    \item $v$ 沿球的半径方向$\frac{\partial v}{\partial r}<0$.
        于是 $\frac{\partial v}{\partial\bm{\nu}} > 0$.
  \end{enumerate}

  作函数
  \[\tilde{u}(M)=\varepsilon v(M)+u(M_0).\]
  在 $M_0$ 点有 $\frac{\partial\tilde{u}}{\partial\bm{\nu}}
  = \varepsilon\frac{\partial v}{\partial\bm{\nu}}>0$,
  令函数
  \[w(M) := u(M)-\tilde{u}(M) = u(M)-\varepsilon v(M)-u(M_0).\]
  
  在区域 $D$ 上考察 $w(M)$:
  \begin{enumerate}[(1)]
    \item $Lw=Lu-\varepsilon Lv-Lu(M_0)=-\varepsilon Lv-cu(M_0)<0$;
    \item 在 $|x| = R/2$ 上由于 $u(M)>u(M_0)$,
      取 $\varepsilon$ 足够小可使得 $w(M)>0$;
    \item 在 $|x| = R$ 上 $v=0$, $u(M)>u(M_0)$, 故 $w(M)\geq 0$.
  \end{enumerate}
  现在证明在整个区域 $D$ 上 $w\geq 0$, 假设存在 $M_1\in D$, 使得 $w(M_1) < 0$, 于是
  \[cw(M_1)\geq 0,\quad \frac{\partial w}{\partial x_i}\bigg|_{M_1} = 0,
    \quad\left(\frac{\partial^2w}{\partial x_i\partial x_j}\right)\bigg|_{M_1}\text{ 非负定}.\]
  又 $a_{ij}=\sum_{r=1}^n g_{ri}g_{rj}$, 故
  \[\sum_{i,j=1}^na_{ij}\frac{\partial^2w}{\partial x_i\partial x_j}
    \bigg|_{M_1}=\sum_{r=1}^n\sum_{i,j=1}^n\frac{\partial^2w}{\partial x_i\partial x_j}
    \bigg|_{M_1}g_{ri}g_{rj}\geq 0,\]
  因此 $Lw|_{M_1}\geq 0$, 与(1)矛盾, 因此在 $D$ 内 $w(M)\geq w(M_0)$, 故
  \[\frac{\partial w}{\partial\bm{\nu}}\bigg|_{M_0} \geq 0.\]
  从而
  \[\frac{\partial u}{\partial\bm{\nu}} > 0. \qedhere\]
\end{proof}
% \setcounter{chapter}{3}
\chapter{二阶线性偏微分方程的分类与总结}

\section{二阶线性方程的分类}


\begin{exercise}
  证明: 两个自变量的二阶线性方程经过自变量的可逆变换后, 其类型不会改变,
  即变换后 $\Delta = a_{12}^2 - a_{11}a_{22}$ 的符号不变.
\end{exercise}

\begin{proof}
  因为
  \[\begin{cases}
  \bar{a}_{11}=a_{11}\xi_x^2+2a_{12}\xi_x\xi_y+a_{22}\xi_y^2, \\
  \bar{a}_{12}=a_{11}\xi_x\eta_x+a_{12}(\xi_x\eta_y+\xi_y\eta_x)+a_{22}\xi_y\eta_y, \\
  \bar{a}_{22}=a_{11}\eta_x^2+2a_{12}\eta_x\eta_y+a_{22}\eta_y^2,
  \end{cases}\]
  所以
  \[\begin{split}
    \overline{\Delta}
    & = \bar{a}_{12}^2-\bar{a}_{11}\bar{a}_{22} \\
    & = a_{12}^2(\xi_x\eta_y+\xi_y\eta_x)^2-4a_{12}^2\xi_x\xi_y\eta_x\eta_y+2a_{11}a_{22}\xi_x\xi_y\eta_x\eta_y-a_{11}a_{22}(\xi_x^2\eta_y^2+\xi_y^2\eta_x^2) \\
    & = (a_{12}^2-a_{11}a_{22})(\xi_x\eta_y-\xi_y\eta_x)^2 \\
    & = \Delta\cdot\left[\frac{D(\xi,\eta)}{D(x,y)}\right]^2.
  \end{split}\]
  故 $\Delta$ 与 $\overline{\Delta}$ 的符号相同.
\end{proof}


\begin{exercise}
  判定下述方程的类型:
  \begin{enumerate}[(1)]
    \item $x^2u_{xx} - y^2u_{yy} = 0$;
    \item $u_{xx} + (x+y)^2 u_{yy} = 0$;
    \item $u_{xx} + xyu_{yy} = 0$;
    \item $u_{xx} - 4u_{xy} + 2u_{xz} + 4u_{yy} + u_{zz} = 0$;
    \item $u_{xx} + (\sgn y)u_{yy} = 0$.
  \end{enumerate}
\end{exercise}

\begin{solution}
  \begin{enumerate}[(1)]
    \item $\Delta = x^2y^2\geq 0$. Parabolic on the axes and hyperbolic in other places;
    \item $\Delta = -(x+y)^2\leq 0$. Parabolic on the line $x+y=0$ and elliptic elsewhere;
    \item $\Delta = -xy$. Elliptic in the 1st and 3rd quadrands, hyperbolic in the 2nd and 4th
      quadrands, and parabolic on the axes;
    \item The corresponding matrix is
      \[\begin{pmatrix}
        1 & -2 & 1 \\
        -2 & 4 & 0 \\
        1 & 0 & 1
      \end{pmatrix}.\]
      Since $D_1=1>0$, $D_2=0$ and $D_3=-4<0$, the type is hyperbolic;
    \item Since
      \[\Delta = -\sgn y \begin{cases}
        <0, & \text{if}\ y>0, \\
        =0, & \text{if}\ y=0, \\
        >0, & \text{if}\ y<0,
      \end{cases}\]
      the type is elliptic when $y>0$, hyperbolic when $y<0$ and parabolic when $y=0$.
  \end{enumerate}
\end{solution}


\begin{exercise}[3]
  化下列方程为标准形式:
  \begin{enumerate}[(1)]
    \item $u_{xx} + 4u_{xy} + 5u_{yy} + u_x + 2u_y = 0$;
    \item $x^2 u_{xx} + 2xy u_{xy} + y^2 u_{yy} = 0$;
    \item $u_{xx} + yu_{yy} = 0$;
    \item $u_{xx} - 2\cos x u_{xy} - (3 + \sin^2 x)u_{yy} - yu_y = 0$;
    \item $(1+x^2)u_{xx} + (1+y^2)u_{yy} + xu_x + yu_y = 0$.
  \end{enumerate}
\end{exercise}

\begin{solve}
  (1) $u_{xx}+4u_{xy}+5u_{yy}+u_x+2u_y=0$.
  $\Delta=4-5=-1<0$, 故方程为椭圆型.
  特征方程为 $\diff y^2-4\diff x\diff y+5\diff x^2=0\Rightarrow\frac{\diff y}{\diff x}=2\pm i\Rightarrow y=(2\pm i)x+C$, 取$y=(2+i)x+C$, 即$y-2x-ix=C$. 令
  \[\begin{cases}
  \xi=2x-y, \\
  \eta=x.
  \end{cases}\]
  则
  \[\begin{cases}
    u_x=2u_{\xi}+u_{\eta}, \\
    u_y=-u_{\xi}, \\
    u_{xx}=2(2u_{\xi\xi}+u_{\xi\eta})+2u_{\xi\eta}+u_{\eta\eta}=4u_{\xi\xi}+4u_{\xi\eta}+u_{\eta\eta}, \\
    u_{yy}=-(-u_{\xi\xi})=u_{\xi\xi}, \\
    u_{xy}=-(2u_{\xi\xi}+u_{\xi\eta}).
  \end{cases}\]
  代入原方程即得标准形式为
  \[u_{\xi\xi}+u_{\eta\eta}+u_{\eta} = 0.\]

  (2) $x^2u_{xx}+2xyu_{xy}+y^2u_{yy}=0$.
  $\Delta=x^2y^2-x^2y^2=0$, 故方程为抛物型.
  特征方程为 $x^2\diff y^2-2xy\diff x\diff y+y^2\diff x^2=0\Rightarrow y=Cx$. 令
  \[\begin{cases}
  \xi = \frac{y}{x}, \\
  \eta = x.
  \end{cases}\]
  则
  \[\begin{cases}
    u_x = -\frac{y}{x^2}u_{\xi}+u_{\eta}, \\
    u_y = \frac{1}{x}u_{\xi}, \\
    u_{xx} = \frac{2y}{x^3}u_{\xi}-\frac{y}{x^2}
      \left(-\frac{y}{x^2}u_{\xi\xi}+u_{\xi\eta}\right)-\frac{y}{x^2}u_{\eta\xi}+u_{\eta\eta}
      = \frac{2y}{x^3}u_{\xi}+\frac{y^2}{x^4}u_{\xi\xi}-\frac{2y}{x^2}u_{\xi\eta}+u_{\eta\eta}, \\
    u_{yy}=\frac{1}{x^2}u_{\xi\xi}, \\
    u_{xy}=-\frac{1}{x^2}u_{\xi}+\frac{1}{x}\left(-\frac{y}{x^2}u_{\xi\xi}+u_{\xi\eta}\right)
      = -\frac{1}{x^2}u_{\xi}-\frac{y}{x^3}u_{\xi\xi}+\frac{1}{x}u_{\xi\eta}.
  \end{cases}\]
  代入原方程即得标准形式为
  \[x^2u_{\eta\eta}=0\Rightarrow u_{\eta\eta} = 0.\]

  (3) $u_{xx}+yu_{yy}=0$.
  $\Delta=-y$, 故 $y>0$时为椭圆型, $y=0$时为抛物型, $y<0$ 时为双曲型.
  特征方程为 $\diff y^2+y\diff x^2=0$.

  (i) $y>0$ 时, $\frac{\diff y}{\diff x}=\pm\sqrt{y}i$,
  取 $\frac{\diff y}{\diff x}=\sqrt{y}i\Rightarrow 2\sqrt{y}-ix=C$,令
  \[\begin{cases}
  \xi = 2\sqrt{y}, \\
  \eta=-x.
  \end{cases}\]
  则
  \[\begin{cases}
    u_x = -u_{\eta}, \\
    u_{xx} = u_{\eta\eta}, \\
    u_{y} = \frac{1}{\sqrt{y}}u_{\xi}, \\
    u_{yy} = -\frac{1}{2}y^{-3/2}u_{\xi}+\frac{1}{y}u_{\xi\xi}.
  \end{cases}\]
  代入原方程即得标准形式为
  \[u_{\eta\eta}+u_{\xi\xi}-\frac{1}{\xi}u_{\xi} = 0.\]

  (ii) $y=0$ 时, $u_{xx}=0$即为标准形式.

  (iii) $y<0$ 时, $\frac{\diff y}{\diff x}=\pm\sqrt{-y}\Rightarrow 2\sqrt{-y}\pm x=C$,令
  \[\begin{cases}
  \xi = 2\sqrt{-y}+x, \\
  \eta = 2\sqrt{-y}-x.
  \end{cases}\]
  则
  \[\begin{cases}
    u_x=u_{\xi}-u_{\eta}, \\
    u_{xx}=u_{\xi\xi}-2u_{\xi\eta}+u_{\eta\eta}, \\
    u_{y}=\frac{-1}{\sqrt{-y}}(u_{\xi}+u_{\eta}), \\
    u_{yy}=-\frac{1}{2}(-y)^{-3/2}(u_{\xi}+u_{\eta})-\frac{1}{y}u_{\xi\xi}-\frac{1}{y}u_{\eta\eta}-\frac{2}{y}u_{\xi\eta}.
  \end{cases}\]
  代入原方程即得标准形式为
  \[u_{\xi\eta}-\frac{1}{2(\xi+\eta)}(u_{\xi}+u_{\eta}) = 0.\]

  (4) $u_{xx}-2\cos xu_{xy}-(3+\sin^2x)u_{yy}-yu_y=0$.
  $\Delta=\cos^2x+3+\sin^2x=4>0$, 故方程为双曲型.
  特征方程为 $\diff y^2+2\cos x\diff x\diff y-(3+\sin^2x)\diff x^2\Rightarrow\frac{\diff y}{\diff x}=-\cos x\pm2\Rightarrow y+\sin x\pm2x=C\Rightarrow y+\sin x\pm 2x=C$.
  令
  \[\begin{cases}
  \xi = y+\sin x+2x, \\
  \eta = y+\sin x-2x.
  \end{cases}\]
  则
  \[\begin{cases}
    u_x = (\cos x+2)u_{\xi}+(\cos x-2)u_{\eta}, \\
    u_y = u_{\xi}+u_{\eta}, \\
    u_{xx} = -\sin x(u_{\xi}+u_{\eta}) + (\cos x+2)^2u_{\xi\xi}+(\cos x-2)^2u_{\eta\eta}
      + 2(\cos^2x-4)u_{\xi\eta}\\
    u_{yy} = u_{\xi\xi}+2u_{\xi\eta}+u_{\eta\eta}, \\
    u_{xy} = (\cos x+2)u_{\xi\xi}+2\cos xu_{\xi\eta}+(\cos x-2)u_{\eta\eta}.
  \end{cases}\]
  代入原方程即得标准形式为
  \[u_{\xi\eta}+\frac{\xi+\eta}{32}(u_{\xi}+u_{\eta}) = 0.\]

  (5) $(1+x^2)u_{xx}+(1+y^2)u_{yy}+xu_x+yu_y=0$.
  $\Delta=-(1+x^2)(1+y^2)<0$,故方程为椭圆型.
  特征方程为$(1+x^2)\diff y^2+(1+y^2)\diff x^2=0\Rightarrow\frac{\diff y}{\diff x}=\pm\sqrt{\frac{1+y^2}{1+x^2}}i\Rightarrow\frac{\diff y}{\sqrt{1+y^2}}=\pm i\frac{\diff x}{\sqrt{1+x^2}}\Rightarrow\ln(y+\sqrt{1+y^2})\pm i\ln(x+\sqrt{1+x^2})=C$\\
  令
  \[\begin{cases}
  \xi = \ln(y+\sqrt{1+y^2}), \\
  \eta=\ln(x+\sqrt{1+x^2}).
  \end{cases}\]
  则
  \[\begin{cases}
    u_x = \frac{1}{\sqrt{1+x^2}}u_{\eta}, \\
    u_y = \frac{1}{\sqrt{1+y^2}}u_{\xi}, \\
    u_{xx} = -x(1+x^2)^{-3/2}u_{\eta}+\frac{1}{1+x^2}u_{\eta\eta}, \\
    u_{yy} = -y(1+y^2)^{-3/2}u_{\xi}+\frac{1}{1+y^2}u_{\xi\xi}.
  \end{cases}\]
  代入原方程即得标准形式为
  \[u_{\xi\xi}+u_{\eta\eta}=0. \qedhere\]
\end{solve}


\begin{exercise}
  证明: 两个自变量的二阶常系数双曲型方程或椭圆型方程一定可以经过自变量
  及未知函数的可逆变换
  \[u = \e^{\lambda\xi + \mu\eta} v\]
  将它化成
  \[v_{\xi\xi} \pm v_{\eta\eta} + cv = f\]
  的形式.
\end{exercise}

\begin{proof}
  已知两个自变量的二阶常系数双曲型方程或椭圆型方程可以通过可逆变换化为标准形式:
  \[u_{\xi\xi}\pm u_{\eta\eta}+au_{\xi}+bu_{\eta}+cu+f=0\]
  下面以椭圆型方程为例,因为$u=e^{\lambda\xi+\mu\eta}v$,所以
  \[\begin{cases}
  u_{\xi}=e^{\lambda\xi+\mu\eta}(\lambda v+v_{\xi})\\
  u_{\eta}=e^{\lambda\xi+\mu\eta}(\mu v+v_{\eta})\\
  u_{\xi\xi}=e^{\lambda\xi+\mu\eta}(v_{\xi\xi}+2\lambda v_{\xi}+\lambda^2v)\\
  u_{\eta\eta}=e^{\lambda\xi+\mu\eta}(v_{\eta\eta}+2\mu v_{\eta}+\mu^2v)
  \end{cases}\]
  故
  \[\begin{split}
  u_{\xi\xi}&+u_{\eta\eta}+au_{\xi}+bu_{\eta}+cu+f\\
  &=e^{\lambda\xi+\mu\eta}\big[v_{\xi\xi}+v_{\eta\eta}+(2\lambda+a)v_{\xi}+(2\mu+b)v_{\eta}+(\lambda^2+\mu^2+a\lambda+b\mu+c)v\big]+f\\
  &=0
  \end{split}\]
  令\[\lambda=-\frac{a}{2},\mu=-\frac{b}{2},c_1=c-\frac{a^2}{4}-\frac{b^2}{4},f_1=-fe^{-(\lambda\xi+\mu\eta)}\]
  即可将原方程化简为
  \[v_{\xi\xi}+v_{\eta\eta}+c_1v=f_1\]
  对于双曲型方程的情形可以进行类似证明.
\end{proof}


\begin{exercise}
  对 $\mathbb{R}^n$ 中诸点判定方程
  \[ \sum_{i,j=1}^n (\delta_{ij}-x_ix_j)\frac{\partial^2 u}{\partial x_i\partial x_j}
      + 2\sum_{i=1}^n x_i \frac{\partial u}{\partial x_i} + cu = f \]
  的类型.
\end{exercise}

\begin{solution}
  The corresponding matrix is
  \[ A = \begin{pmatrix}
          1-x_1^2 & -x_1x_2 & \cdots & -x_1x_n \\
          -x_2x_n & 1-x_2^2 & \cdots & -x_2x_n \\
          \vdots  & \vdots  & \ddots & \vdots  \\
          -x_1x_n & -x_2x_n & \cdots & -x_n^2
        \end{pmatrix}
       = E - xx^T,
  \]
  where $x = (x_1,\ldots,x_n)^T$. By \href{https://en.wikipedia.org/wiki/Matrix_determinant_lemma}{matrix determinant lemma}, we have for $\lambda\neq 1$,
  \begin{align*}
    |\lambda E-A|
    & = |(\lambda-1)E + xx^T| \\
    & = \biggl(1 + x^T\frac{E}{\lambda-1}x\biggr) \det \bigl((\lambda-1)E\bigr) \\
    & = (\lambda-1)^{n-1} (\lambda-1+x^Tx).
  \end{align*}
  Obviously the case $\lambda=1$ can also be incorporated into the above formula.
  Therefore we find that the eigenvalues of $A$ are $\lambda_1=1$
  (with multiplicity $n-1$ or $n$)
  and $\lambda_2 = 1-x^Tx$. So
  \begin{itemize}
    \item If $x^Tx = \sum x_i^2 =1$, the equation is parabolic;
    \item If $x^Tx = \sum x_i^2 <1$, the equation is elliptic;
    \item If $x^Tx = \sum x_i^2 >1$, the equation is hyperbolic. \qedhere
  \end{itemize}
\end{solution}


\section{二阶线性方程的特征理论}


\begin{exercise}
  求下列方程的特征方程和特征方向:
  \begin{enumerate}[(1)]
    \item $\frac{\partial^2 u}{\partial x_1^2} + \frac{\partial^2 u}{\partial x_2^2} = \frac{\partial^2 u}{\partial x_3^2} + \frac{\partial^2 u}{\partial x_4^2}$,
    \item $\frac{\partial^2 u}{\partial t^2} = \sum_{i=1}^3 \frac{\partial^2 u}{\partial x_i^2}$,
    \item $\frac{\partial u}{\partial t} = \frac{\partial^2 u}{\partial x^2} - \frac{\partial^2 u}{\partial y^2}$.
  \end{enumerate}
\end{exercise}


\begin{exercise}
  对波动方程 $u_{tt} - a^2(u_{xx}+u_{yy}) = 0$, 求过直线 $l: t = 0, y = 2x$ 的特征平面.
\end{exercise}

\begin{solve}
  The characteristic equation is
  \[\alpha_0^2-a^2(\alpha_1^1+\alpha_2^2) = 0.\]
  Combining $\alpha_0^2+\alpha_1^2+\alpha_2^2=1$, we find
  \[\alpha_0 = \pm\frac{a}{\sqrt{1+a^2}},
    \quad \alpha_1=\frac{\cos\theta}{\sqrt{1+a^2}},
    \quad \alpha_2=\frac{\sin\theta}{\sqrt{1+a^2}}.\]
  Since the characteristic hyperplane passes through the origin,
  we may write its equation as
  \begin{equation}\label{eq:4.6}
    at + \cos\theta\, x + \sin\theta\, y = 0.
  \end{equation}
  On the other hand, the line $t=0, y=2x$ is on the hyperplane \eqref{eq:4.6}.
  So $\cos\theta+2\sin\theta = 0$, from which we get
  $\cos\theta = \mp\frac{2\sqrt{5}}{5}$ and $\sin\theta = \pm\frac{\sqrt{5}}{5}$.
  Thus the hyperplane is
  \[ at \mp \frac{2\sqrt{5}}{5}x \pm \frac{\sqrt{5}}{5}y = 0. \qedhere \]
\end{solve}


\begin{exercise}
  证明: 经过可逆的坐标变换 $x_i = f_i(y_1, \ldots, y_n)$ $(i=1, \ldots, n)$,
  原方程的特征曲面变为经变换后的新方程的特征曲面,
  即特征曲面关于可逆坐标变换具有不变形.
\end{exercise}

\begin{proof}
  考虑二阶线性方程
  \begin{equation}\label{eq:4.7}
    \sum_{i,j=1}^nA_{ij}\frac{\partial^2u}{\partial x_i\partial x_j}
    + \sum_{i=1}^nB_i\frac{\partial u}{\partial x_i} + Cu = F.
  \end{equation}
  设 $G(x_1,\ldots,x_n)=0$ 为其特征曲面, 则
  \begin{equation}\label{eq:4.1}
    \sum_{i,j=1}^nA_{ij}\frac{\partial G}{\partial x_i}\frac{\partial G}{\partial x_j}=0.
  \end{equation}
  经过可逆的坐标变换 $x_i=f_i(y_1,\ldots,y_n)$, 有
  \[\frac{\partial u}{\partial x_i}
    = \sum_{l=1}^n\frac{\partial u}{\partial y_l}\frac{\partial y_l}{\partial x_i},\]
  \[\frac{\partial^2u}{\partial x_i\partial x_j}
    = \sum_{k,l=1}^n\frac{\partial^2u}{\partial y_l\partial y_k}\frac{\partial y_l}{\partial x_i}\frac{\partial y_k}{\partial x_j}
      + \sum_{l=1}^n\frac{\partial u}{\partial y_l}\frac{\partial^2y_l}{\partial x_i\partial x_j}.\]
  将上面两式代入原方程~\eqref{eq:4.7} 得
  \[ \sum_{i,j=1}^nA_{ij}\Biggl(\sum_{k,l=1}^n\frac{\partial^2u}{\partial y_l\partial y_k}\frac{\partial y_l}{\partial x_i}\frac{\partial y_k}{\partial x_j}
    + \sum_{l=1}^n\frac{\partial u}{\partial y_l}\frac{\partial^2y_l}{\partial x_i\partial x_j}\Biggr)
    + \sum_{i=1}^nB_i\Biggl(\sum_{l=1}^n\frac{\partial u}{\partial y_l}\frac{\partial y_l}{\partial x_i}\Biggr) + Cu = F. \]
  整理上式并简记一阶偏导数项得
  \[ \sum_{k,l=1}^n\Biggl(\sum_{i,j=1}^nA_{ij}\frac{\partial y_l}{\partial x_i}\frac{\partial y_k}{\partial x_j}\Biggr)\frac{\partial^2u}{\partial y_l\partial y_k}
    + \sum_{l=1}^n\widetilde{B}_l\frac{\partial u}{\partial y_l} + Cu = F. \]
  设 $G^*(y_1,\ldots,y_n)$ 为其特征曲面, 则需满足
  \[ \sum_{k,l=1}^n\Biggl(\sum_{i,j=1}^nA_{ij}\frac{\partial y_l}{\partial x_i}\frac{\partial y_k}{\partial x_j}\Biggr)\frac{\partial G^*}{\partial y_k}\frac{\partial G^*}{\partial y_l} = 0. \]
  另一方面, 对原方程的特征曲面经过可逆变换后的特征曲面为:
  \[ G(x_1,\ldots,x_n) = G(f_1(y_1,\ldots,y_n),\ldots,f_n(y_1,\ldots,y_n)) =: G_1(y_1,\ldots,y_n), \]
  由 \eqref{eq:4.1} 得
  \[ \sum_{i,j=1}^nA_{ij}\Biggl(\sum_{l=1}^n\frac{\partial G_1}{\partial y_l}\frac{\partial y_l}{\partial x_i}\Biggr)\Biggl(\sum_{k=1}^n\frac{\partial G_1}{\partial y_k}\frac{\partial y_k}{\partial x_j}\Biggr)
    = \sum_{k,l=1}^n\Biggl(\sum_{i,j=1}^nA_{ij}\frac{\partial y_l}{\partial x_i}\frac{\partial y_k}{\partial x_j}\Biggr)\frac{\partial G_1}{\partial y_k}\frac{\partial G_1}{\partial y_l}=0. \]
  对比即得$G^*=G_1$,即特征曲面关于可逆坐标变换具有不变性.
\end{proof}


\begin{exercise}
  试证二阶线性偏微分方程解的 $m$ 阶弱间断 (即直至 $m-1$ 阶的偏导数为连续, 而 $m$ 阶偏导数为第一类间断)
  也只可能沿着特征线发生.
\end{exercise}

\begin{proof}
  Omit. Similar to the proof in the textbook.
\end{proof}


\begin{exercise}
  试定义 $n$ 阶线性偏微分方程的特征线、特征方向和特征曲面.
\end{exercise}

\begin{solution}
  For $n$-th order linear partial differential equation
  \begin{equation}\label{eq:4.2}
    \sum_{|\alpha|\leq n} a_\alpha(x) D^\alpha u(x) = 0,
  \end{equation}
  where $\alpha=(\alpha_1,\ldots,\alpha_n)$ is a multi-index
  and $D^\alpha u(x) = \frac{\partial^{|\alpha|} u}{\partial x_1^{\alpha_1}\cdots\partial x_n^{\alpha_n}}$,
  $|\alpha| = \sum \alpha_i$. We say that the surface
  \begin{equation}\label{eq:4.3}
    S: \varphi(x) = 0
  \end{equation}
  is the characteristic surface of \eqref{eq:4.1} if
  \begin{equation}\label{eq:4.4}
    \sum_{|\alpha|=n} a_\alpha(x) (\nabla\varphi(x))^\alpha
      = \sum_{|\alpha|=n} a_\alpha(x) \prod_{i=1}^n
        \biggl(\frac{\partial\varphi}{\partial x_i}\biggr)^{\alpha_i} = 0.
  \end{equation}
  And a vector $v = (v_1,\ldots,v_n)$ is called the characteristic vector
  if it satisfies the following characteristic equation
  \begin{equation}\label{eq:4.5}
    \sum_{|\alpha|=n} a_\alpha(x) v^\alpha
      = \sum_{|\alpha|=n} a_\alpha(x) v_1^{\alpha_1} \cdots v_n^{\alpha_n} = 0.
  \end{equation}
\end{solution}


\section{三类方程的比较}


\begin{exercise}
  证明热传导方程
  \[\frac{\partial u}{\partial t} = a^2 \frac{\partial^2 u}{\partial x^2}\]
  的初边值问题
  \[\begin{cases}
    u(0, t) = u(l, t) = 0, \\
    u(x, 0) = \varphi(x)
  \end{cases}\]
  的解关于自变量 $x$ ($0<x<l$) 和 $t$ ($t>0$) 可进行任意次微分.
\end{exercise}

\begin{proof}
  利用分离变量法得该初边值问题的解为
  \[u(x,t) = \sum_{n=1}^{\infty}C_ne^{-\frac{n^2\pi^2a^2}{l^2}t}\sin\frac{n\pi}{l}x,\]
  其中 $C_n=\frac{2}{l}\int_0^l\varphi(x)\sin\frac{n\pi}{l}x\diff x$,
  $|C_n|\leq M$, 只需要证明级数逐项微分任意次后仍然是绝对且一致收敛即可,
  对 $t$ 微分 $\alpha$ 次, 对 $x$ 微分 $\beta$ 次需要级数
  \[\sum_{n=1}^{\infty}C_n\left(-\frac{n^2\pi^2a^2}{l^2}\right)^{\alpha}\left(\frac{n\pi}{l}\right)^{\beta}\left(\sin\frac{n\pi}{l}x\right)^{(\beta)}e^{-\frac{n^2\pi^2a^2}{l^2}t}\]
  绝对且一致收敛,而当$t\geq t_0>0$时,上述级数以
  \[\sum_{n=1}^{\infty}M\left(\frac{n\pi a}{l}\right)^{2\alpha}\left(\frac{n\pi}{l}\right)^{\beta}e^{-\frac{n^2\pi^2a^2}{l^2}t_0}\]
  为优级数,易知此级数收敛,故原级数绝对且一致收敛.
\end{proof}


\section{先验估计}

\begin{exercise}
  设 $u(x_1,\cdots,x_n)$ 在区域 $\Omega$ 上非负, 且满足不等式
  \[\sum_{i,j=1}^n a_{ij}(x) u_{x_ix_j} + \sum_{i=1}^n b_i(x) u_{x_i} + c(x)u\geq 0,\]
  其中 $a_{ij}$, $b_i$, $c$ 在 $\overline{\Omega}$ 上具有一阶连续偏导数,
  满足 (4.38) 式, 且 $c(x)\leq 0$,
  证明极值原理 $\max_{\overline{\Omega}}u=\max_{\Gamma}u$ 成立.
\end{exercise}

\begin{proof}
  See \emph{Elliptic Partial Differential Equations} (Han Qing \& Lin Fanghua) 
  Lemma 2.1 and Theorem 2.3.
\end{proof}


\begin{exercise}[3]
  在 $Q_T = (0,l)\times (0,T)$ 中考察下列初边值问题
  \begin{align*}
    & u_{tt} - a^2 u_{xx} + b(x,t)u_x + b_0(x,t)u_x + c(x,t)u = f(x,t), \\
    & u|_{x=0} = 0,\qquad (u_x + ku)|_{x=l} = 0, \\
    & u|_{t=0} = \varphi(x), \qquad u_t|_{t=0} = \psi(x),
  \end{align*}
  证明其解的唯一性及稳定性.
\end{exercise}

\begin{proof}
  令
\end{proof}


\begin{exercise}
  建立下列初边值问题的能量估计式:
  \[u_t - \Delta u + \sum_{i=1}^n b_i(x,t) u_{x_i} + c(x,t)u = f(x,t),\]
  \[\frac{\partial u}{\partial n}\bigg|_{\varGamma} = 0,
    \qquad u|_{t=0} = \varphi(x).\]
\end{exercise}

\begin{proof}
  任取 $T>0$,下面在$[0,T]$上建立能量估计式, 记$E(t)=\int_{\Omega}u^2(x,t)\diff x$,
  则$E'(t)=2\int_{\Omega}uu_t\diff x$,代入原方程得
  \[E'(t)=2\int_{\Omega}u\Delta u\diff x-2\sum_{i=1}^n\int_{\Omega}b_iuu_{x_i}\diff x-2\int_{\Omega}cu^2\diff x+2\int_{\Omega}uf\diff x\]
  由格林公式及边界条件得
  \[\begin{split}
  \int_{\Omega}u\Delta u\diff x&=\int_{\Omega}\sum_{k=1}(uu_{x_k})_{x_k}\diff x-\int_{\Omega}|\nabla u|^2\diff x\\
  &=\int_{\partial\Omega}\sum_{k=1}^nuu_{x_k}\cos(\vec{n},x_k)\diff x-\int_{\Omega}|\nabla u|^2\diff x\\
  &=\int_{\partial\Omega}u\frac{\partial u}{\partial\vec{n}}\diff x-\int_{\Omega}|\nabla u|^2\diff x=-\int_{\Omega}|\nabla u|^2\diff x
  \end{split}\]
  再设$|b(x,t)|,|c(x,t)|$在$\overline{R}_T=\overline{\Omega}\times[0,T]$上的最大值为$M$,记$C_0=\frac{M}{2}\max(1,\frac{1}{a^2})$,利用加权平均值不等式$2ab\leq\epsilon a^2+\frac{1}{\epsilon}b^2(\epsilon>0)$得
  \[\begin{split}
  E'(t)&=-2\int_{\Omega}|\nabla u|^2\diff x+2M\sum_{i=1}^n\int_{\Omega}|uu_{x_i}|\diff x+2M\int_{\Omega}u^2\diff x+2\int_{\Omega}|uf|\diff x\\
  &\leq-2\int_{\Omega}|\nabla u|^2\diff x+M\left(\epsilon\int_{\Omega}|\nabla u|^2\diff x+\frac{n}{\epsilon}\int_{\Omega}u^2\diff x\right)+(2M+1)\int_{\Omega}u^2\diff x+\int_{\Omega}f^2\diff x
  \end{split}\]
  取$\epsilon=\frac{1}{M}$,并记$\widehat{C}=nM^2+2M+1$,则
  \[\begin{split}
  E'(t)&\leq-\int_{\Omega}|\nabla u|^2\diff x+(nM^2+2M+1)\int_{\Omega}u^2\diff x+\int_{\Omega}f^2\diff x\\
  &\leq-\int_{\Omega}|\nabla u|^2\diff x+\widehat{C}\int_{\Omega}u^2\diff x+\int_{\Omega}f^2\diff x
  \end{split}\]
  由Gronwall不等式得
  \[\begin{split}
  E(t)&\leq e^{\widehat{C}t}E(0)-\int_0^te^{\widehat{C}(t-s)}\diff s\int_{\Omega}|\nabla u|^2\diff x+\int_0^te^{\widehat{C}(t-s)}\diff s\int_{\Omega}f^2\diff x\\
  &\leq e^{\widehat{C}t}E(0)-\int_0^t\diff s\int_{\Omega}|\nabla u|^2\diff x+\int_0^te^{\widehat{C}(t-s)}\diff s\int_{\Omega}f^2\diff x
  \end{split}\]
  故
  \[\int_{\Omega}u^2(x,t)\diff x+\int_0^t\diff s\int_{\Omega}|\nabla u|^2\diff x\leq e^{\widehat{C}t}\left(\int_{\Omega}\varphi^2(x)\diff x+\int_0^t\diff s\int_{\Omega}f^2\diff x\right)\]
\end{proof}


\begin{exercise}
  考察初边值问题
  \begin{align*}
    & \Delta u + \sum_{i=1}^{n} b_i(x) u_{x_i} + c(x)u = f, \\
    & \frac{\partial u}{\partial n}\bigg|_{\varGamma} = 0.
  \end{align*}
  试证当 $c(x)$ 充分负时, 其解在能量模意义下的稳定性.
\end{exercise}

\begin{proof}
  在方程两边同时乘以 $u$ 并在 $\Omega$ 上积分得
  \[\int_{\Omega}fu\diff x 
    = \int_{\Omega}\left(cu^2+\sum_{i=1}^nb_iu_{x_i}u+u\Delta u\right)\diff x.\]
  利用格林公式及边界条件得
  \[\begin{split}
  \int_{\Omega}u\Delta u\diff x
  &=\int_{\Omega}\sum_{k=1}^n(uu_{x_k})_{x_k}\diff x-\int_{\Omega}|\nabla u|^2\diff x\\
  &=\int_{\partial\Omega}\sum_{k=1}^nuu_{x_k}\cos(\vec{n},x_k)\diff x-\int_{\Omega}|\nabla u|^2\diff x\\
  &=\int_{\partial\Omega}u\frac{\partial u}{\partial\vec{n}}\diff x-\int_{\Omega}|\nabla u|^2\diff x=-\int_{\Omega}|\nabla u|^2\diff x
  \end{split}\]
  故
  \[\int_{\Omega}fu\diff x\leq-\int_{\Omega}|\nabla u|^2\diff x+\int_{\Omega}\sum b_iu_{x_i}u\diff x+\int_{\Omega}cu^2\diff x\]
  记$M=\max_{1\leq i\leq n}\max_{x\in\Omega}|b_i(x)|$,则
  \[\begin{split}
  \int_{\Omega}|\nabla u|^2\diff x-\int_{\Omega}cu^2\diff x
  &\leq\sum_{i=1}^n\int_{\Omega}b_iu_{x_i}u\diff x-\int_{\Omega}fu\diff x\\
  &\leq2M\int_{\Omega}\sum_{i=1}^n|u_{x_i}u|\diff x+\int_{\Omega}|fu|\diff x\\
  &\leq2M\left(\frac{\epsilon}{2}\int_{\Omega}|\nabla u|^2\diff x+\frac{n}{2\epsilon}\int_{\Omega}u^2\diff x\right)+\int_{\Omega}\left(\frac{1}{2}u^2+\frac{1}{2}f^2\right)\diff x
  \end{split}\]
  取$\epsilon=\frac{1}{2M}$,则
  \[\int_{\Omega}|\nabla u|^2\diff x-\int_{\Omega}cu^2\diff x\leq\frac{1}{2}\int_{\Omega}|\nabla u|^2\diff x+\left(2nM+\frac{1}{2}\right)\int_{\Omega}u^2\diff x+\frac{1}{2}\int_{\Omega}f^2\diff x\]
  令$\gamma_0=2nM^2+1$,则当$c(x)\leq-\gamma_0$时,有
  \[\int_{\Omega}|\nabla u|^2\diff x+\left(2nM^2+1\right)\int_{\Omega}u^2\diff x\leq\int_{\Omega}|\nabla u|^2\diff x-\int_{\Omega}cu^2\diff x\]
  将上面两式结合,得
  \[\int_{\Omega}|\nabla u|^2\diff x+(2nM^2+1)\int_{\Omega}u^2\diff x\leq\frac{1}{2}\int_{\Omega}|\nabla u|^2\diff x+\left(2nM^2+\frac{1}{2}\right)\int_{\Omega}u^2\diff x+\frac{1}{2}\int_{\Omega}f^2\diff x\]
  所以
  \[\frac{1}{2}\int_{\Omega}|\nabla u|^2\diff x+\frac{1}{2}\int_{\Omega}u^2\diff x\leq\frac{1}{2}\int_{\Omega}f^2\diff x\Rightarrow\int_{\Omega}(|\nabla u|^2+u^2)\diff x\leq C\int_{\Omega}f^2\diff x\]
\end{proof}
% \chapter{一阶偏微分方程组}

\section{引言}

\begin{exercise}
  把波动方程
  \[\frac{\partial^2u}{\partial t^2} = a^2 
    \biggl(\frac{\partial^2u}{\partial x^2} + \frac{\partial^2u}{\partial y^2}
    + \frac{\partial^2u}{\partial z^2}\biggr)\]
  带初始条件
  \[\begin{cases}
    u|_{t=0} = \varphi(x,y,z), \\
    \frac{\partial u}{\partial t}|_{t=0} = \psi(x,y,z)
  \end{cases}\]
  的柯西问题化为一个一阶方程组的柯西问题, 并证明其解的等价性.
\end{exercise}

\begin{proof}
  令 $p=\frac{\partial u}{\partial t}$, $q_1=\frac{\partial u}{\partial x}$,
  $q_2=\frac{\partial u}{\partial y}$, $q_3=\frac{\partial u}{\partial z}$, 则
  \[\begin{cases}
    \frac{\partial p}{\partial t} = a^2\left(\frac{\partial q_1}{\partial x}
      +\frac{\partial q_2}{\partial y}+\frac{\partial q_3}{\partial z}\right) & (1) \\
    \frac{\partial p}{\partial x} = \frac{\partial q_1}{\partial t},
    \frac{\partial p}{\partial y}=\frac{\partial q_2}{\partial t},
    \frac{\partial p}{\partial z}=\frac{\partial q_3}{\partial t} & (2) \\
    t = 0: p=\psi, q_1 = \frac{\partial\varphi}{\partial x},
                   q_2 = \frac{\partial\varphi}{\partial y},
                   q_3 = \frac{\partial\varphi}{\partial z} & (3)
  \end{cases}\]
  原方程的解显然满足新方程,而如果新方程的解为 $(p,q_1,q_2,q_3)$,则
  \[u(x,y,z,t) = \varphi(x,y,z)
    + \int_{(0,0,0,0)}^{(x,y,x,t)}p\diff t+q_1\diff x+q_2\diff y+q_3\diff z\]
  是原方程的解, 其中条件 (2) 确保了积分与路径无关, 故积分定义是合理的.
\end{proof}


\begin{exercise}
  把方程
  \[ u_{tt} = u_{x}^2 + u_{y}^2 \]
  带初始条件
  \[\begin{cases}
    u|_{t=0} = 0, \\
    u_t|_{t=0} = e^x \sin y
  \end{cases}\]
  的柯西问题化为一个一阶偏微分方程组的柯西问题.
\end{exercise}

\begin{solution}
  Let $p=u_t$, then
  \[\begin{cases}
    p = u_t, \\
    p_t = u_x^2 + u_y^2.
  \end{cases}\]
  And the initial value condition is
  \[\begin{cases}
    u|_{t=0} = 0, \\
    p|_{t=0} = e^x \sin y.
  \end{cases}\qedhere\]
\end{solution}


\begin{exercise}
  证明柯瓦列夫斯卡娅型方程 (1.9) 满足初始条件
  \begin{equation}
    t=0: u=\varphi_0(x),\ldots, \frac{\partial^{m-1}u}{\partial t^{m-1}} = \varphi_{m-1}(x) \tag{$\star$}
  \end{equation}
  的柯西问题可以化为一阶方程组的柯西问题, 并证明其解的等价性.
\end{exercise}

\begin{proof}
  Let $\alpha = (\alpha_0,\alpha_1,\ldots,\alpha_n)$ be the general multi-index
  and let 
  \begin{align*}
    \alpha^0 & := (\alpha_0+1,\alpha_1,\ldots,\alpha_n), \\
    \alpha^i & := (\alpha_0,\alpha_1,\ldots,\alpha_i+1,\ldots,\alpha_n).
  \end{align*}
  Decompose $\alpha$ as follows:
  \[ \alpha = \beta + \gamma, \]
  where $\beta=(\alpha_0,0,\ldots,0)$ and $\gamma=(0,\alpha_1,\ldots,\alpha_n)$.
  We denote the special multi-index $\alpha^* = (m-1,0,\ldots,0)$
  and introduce new functions by $u_\alpha = D^\alpha u$.
  Then the Kowalevskaya type equation can be transformed into
  the following first-order system
  \begin{equation}\label{eq:5.1}
    \begin{cases}
      \frac{\partial u_\alpha}{\partial t} = u_{\alpha^0}, \quad |\alpha|\leq m-1, \alpha_0\leq m-2, \\
      \frac{\partial u_\alpha}{\partial x_i} = u_{\alpha^i}, \quad |\alpha|\leq m-1, \\
      \frac{\partial u_{\alpha^*}}{\partial t} = F(t,x,u,u_\alpha,|\alpha|\leq m),
    \end{cases}
  \end{equation}
  together with the following initial value conditions
  \begin{equation}\label{eq:5.2}
    \begin{cases}
      u|_{t=0} = \varphi_0(x), \\
      u_\alpha|_{t=0} = D^\gamma \varphi_{\alpha_0}(x),\quad 1\leq |\alpha|\leq m, \alpha_0\leq m-1. 
    \end{cases}
  \end{equation}

  Now we prove that equivalence of solutions. First of all,
  if $u$ is a solution to (1.9) in the textbook with initial value condition $(\star)$,
  it is straightforward to verify that $(u_\alpha)$ is the solution to~\eqref{eq:5.1} 
  and~\eqref{eq:5.2}.

  Conversely, let $(u_\alpha)$ be the solution to~\eqref{eq:5.1} and \eqref{eq:5.2}.
  Then for all $|\alpha|\leq m-1$ with $\alpha_0\leq m-2$ we have
  \begin{equation}\label{eq:5.3}
    \begin{cases}
      \frac{\partial u_\alpha}{\partial t} = u_{\alpha^0}, \\
      \frac{\partial u_{\alpha^0}}{\partial x_i} = \frac{\partial u_{\alpha^i}}{\partial t}.
    \end{cases}
  \end{equation}
  It follows that
  \begin{equation}\label{eq:5.4}
    \frac{\partial}{\partial t}\biggl(u_{\alpha^i} - \frac{\partial u_\alpha}{\partial x_i}\biggr)
      = 0.
  \end{equation}
  So $u_{\alpha^i} - \frac{\partial u_\alpha}{\partial x_i}$ is independent of $t$.
  From \eqref{eq:5.2} we know that it is equal to zero at $t=0$, thus
  \begin{equation}\label{eq:5.5}
    u_{\alpha^i} \equiv \frac{\partial u_\alpha}{\partial x_i} \quad \text{for all}\ t\geq 0.
  \end{equation}
  Plugging \eqref{eq:5.5} into \eqref{eq:5.1} we find that $u$ is a solution to (1.9)
  in the textbook with the given initial value condition.
\end{proof}

\section{两个自变量的一阶线性偏微分方程组的特征理论}

\begin{exercise}
  求一阶方程
  \begin{enumerate}[(1)]
    \item $\frac{\partial u}{\partial t} + a(x,t) \frac{\partial u}{\partial x} 
      b(x,t)u + c(x,t) = 0$,
    \item $\frac{\partial u}{\partial t} + a(x,t) \frac{\partial u}{\partial x}
      + b(x,t,u) = 0$
  \end{enumerate}
  的特征线和解沿特征线应成立的关系式.
\end{exercise}

\begin{solve}
  (1) 特征线满足的方程为 $\frac{\diff x}{\diff t} = a(x,t)$,
  解得特征线为 $x = \int_0^t a(x,\tau) \diff\tau$.
  在特征线上, $u(x,t) = u\Bigl(\int_0^t a(x, \tau) \diff\tau, t\Bigr)$, 故
  \[\frac{\diff u}{\diff t} = \frac{\partial u}{\partial x} a(x,t)
    + \frac{\partial u}{\partial t}.\]
  故 $u$ 在特征线上满足关系式
  \[\frac{\diff u}{\diff t} + b(x,t)u + c(x,t) = 0.\]

  (2) 同理 $\frac{\diff u}{\diff t} + b(x,t,u) = 0$.
\end{solve}


\begin{exercise}
  求下列一阶方程带初始条件 $u|_{t=0} = \varphi(x)$ 的柯西问题的解:
  \begin{enumerate}[(1)]
    \item $\frac{\partial u}{\partial t} + \frac{\partial u}{\partial x} = 0$;
    \item $\frac{\partial u}{\partial t} + \frac{\partial u}{\partial x} = u$.
  \end{enumerate}
\end{exercise}

\begin{solve}
  (1) 特征线为 $x = t+C$, 在特征线上, $\frac{\diff u}{\diff t} = 
    \frac{\partial u}{\partial x} + \frac{\partial u}{\partial t} = 0$,
  故 $u$ 在特征线上为常数. 故
  \[u(x_0, t_0) = u(x_0-t_0, 0) = \varphi(x_0-t_0).\]
  因此 $u(x,t) = \varphi(x-t)$.

  (2) 特征线为 $x = t+C$, 在特征线上, $\frac{\diff u}{\diff t} = u$,
  故 $u = C\e^t$, 令 $t=0$, 得 $u|_{t=0} = C$. 故
  \[u(x_0, t_0) = u(x_0-t_0, 0) \e^{t_0} = \varphi(x_0-t_0)\e^{t_0}.\]
  因此
  \[u(x, t) = \varphi(x-t) \e^t. \qedhere\]
\end{solve}


\begin{exercise}
  判断方程组
  \begin{align*}
    & \frac{\partial u_1}{\partial t} = a(x,t) \frac{\partial u_1}{\partial x}
      - b(x,t) \frac{\partial u_2}{\partial x} + f_1, \\
    & \frac{\partial u_2}{\partial t} = b(x,t) \frac{\partial u_1}{\partial x}
      + a(x,t) \frac{\partial u_2}{\partial x} + f_2
  \end{align*}
  属于何种类型.
\end{exercise}

\begin{solve}
  \[A = \begin{pmatrix}
    a & -b \\
    b & a
  \end{pmatrix} \Rightarrow (\lambda-a)^2+b^2 = 0.\]
  故当 $b=0$ 时为双曲型, 当 $b\neq 0$ 时为椭圆型.
\end{solve}


\begin{exercise}
  将下列各方程组化为对角型方程组:
  \begin{enumerate}[(1)]
    \item $\begin{cases}
             \frac{\partial u}{\partial t} + (1+\sin x)\frac{\partial u}{\partial x}
                + 2 \frac{\partial v}{\partial x} + x = 0, \\
             \frac{\partial v}{\partial t} + u = 0;  
           \end{cases}$
    \item $\begin{cases}
             \frac{\partial u}{\partial t} = x \frac{\partial u}{\partial x}
               + \frac{\partial v}{\partial x}, \\
             \frac{\partial v}{\partial t} = a^2 \frac{\partial u}{\partial x}
               + x \frac{\partial v}{\partial x}\quad (a>0);
           \end{cases}$
    \item $\begin{cases}
             \frac{\partial u_1}{\partial t} + 6 \frac{\partial u_1}{\partial x}
               + 5 \frac{\partial u_2}{\partial x} = 0, \\
             \frac{\partial u_2}{\partial t} + 5 \frac{\partial u_1}{\partial x}
               + 6 \frac{\partial u_2}{\partial x} = 2u_1, \\
             3 \frac{\partial u_3}{\partial t} + 6 \frac{\partial u_3}{\partial x}
               - 3 \frac{\partial u_1}{\partial x} = 2u_2 + 3u_3 - 3u_1.
           \end{cases}$
  \end{enumerate}
\end{exercise}

\begin{solve}
  (1) \[A = \begin{pmatrix}
  1+\sin x&2\\0&0\end{pmatrix}\Rightarrow\lambda=0\text{\ 或\ }\lambda=1+\sin x.\]
  相应的特征向量为$(-2,1+\sin x)^T,(c,0)^T$,作变换
  \[\begin{cases}
  u=-2v_1+cv_2\\
  v=(1+\sin x)v_1
  \end{cases}\]
  则得
  \[\begin{cases}
  (1+\sin x)\frac{\partial v_1}{\partial t}-2v_1+cv_2=0, \\
  c(1+\sin x)\frac{\partial v_2}{\partial t}+c(1+\sin x)^2\frac{\partial v_2}{\partial x}
    + x(1+\sin x)-4v_1+2cv_2=0.
  \end{cases}\]

  (2) \[A=\begin{pmatrix}
        x & 1 \\ a^2 & x
  \end{pmatrix}\Rightarrow(\lambda-x)^2-a^2=0\Rightarrow\lambda=x\pm a\]
  对应特征向量为 $(-1,a)^T$, $(1,a)^T$,作变换
  \[\begin{cases}
  u=-v_1+v_2\\
  v=av_1+av_2
  \end{cases}\]
  则得
  \[\begin{cases}
  \frac{\partial v_1}{\partial t}=(x-a)\frac{\partial v_1}{\partial x}\\
  \frac{\partial v_2}{\partial t}=(x+a)\frac{\partial v_2}{\partial x}
  \end{cases}\]

  (3)\[A=
  \begin{pmatrix}
  6&5&0\\5&6&0\\-1&0&2
  \end{pmatrix}\Rightarrow
  \begin{vmatrix}
  \lambda-6&-5&0\\-5&\lambda-6&0\\1&0&\lambda-2
  \end{vmatrix}=0.\]
  解得 $(\lambda-2)(\lambda^2-12\lambda+11)=0\Rightarrow\lambda=1,2,11$,
  对应特征向量为 $(1,-1,1)^T$, $(0,0,1)^T$, $(9,9,-1)^T$,作变换
  \[\begin{cases}
  u_1=v_1+9v_3\\
  u_2=-v_1+9v_3\\
  u_3=v_1+v_2-v_3
  \end{cases}\]
  \[R=\begin{pmatrix}
  1&0&9\\-1&0&9\\1&1&-1
  \end{pmatrix}
  \Rightarrow
  R^{-1}=
  \begin{pmatrix}
  1/2&-1/2&0\\-4/9&5/9&1\\1/18&1/18&0
  \end{pmatrix}\]
  又因为
  \[B=\begin{pmatrix}
  0&0&0\\-2&0&0\\1&-2/3&-1
  \end{pmatrix},C=\begin{pmatrix}
  0\\0\\0
  \end{pmatrix}\]
  所以
  \[R^{-1}BR=\begin{pmatrix}
  1&0&9\\-4/9&-1&-6\\-1/9&0&-1
  \end{pmatrix},R^{-1}C=\begin{pmatrix}
  0\\0\\0
  \end{pmatrix}\]
  即得对角型方程组
  \[\begin{cases}
  \frac{\partial v_1}{\partial t}+\frac{\partial v_1}{\partial x}+v_1+9v_3=0\\
  \frac{\partial v_2}{\partial t}+2\frac{\partial v_2}{\partial x}-\frac{4}{9}v_1-v_2-6v_3=0\\
  \frac{\partial v_3}{\partial t}+11\frac{\partial v_3}{\partial x}-\frac{1}{9}v_1-v_3=0
  \end{cases}\]
\end{solve}


\begin{exercise}
  证明: 经过未知函数的任何实系数的可逆线性变换, 方程组 (2.1) 在每一点的特征线方向
  (或特征曲线) 保持不变, 因此也不会改变方程组 (2.1) 所属的类型.
\end{exercise}

\begin{proof}
  原方程为
  \[\frac{\partial U}{\partial t}+A\frac{\partial U}{\partial x}+BU+C = 0.\]
  未知函数作可逆线性变换 $U=RV$ 后, 有
  \[R\frac{\partial V}{\partial t}+AR\frac{\partial V}{\partial x}
    + \left(\frac{\partial R}{\partial t}+A\frac{\partial R}{\partial x}
    + BR\right)V + C = 0.\]
  两端左乘 $R^{-1}$ 得
  \[\frac{\partial V}{\partial t} + A' \frac{\partial V}{\partial x}
    + R^{-1}\biggl(\frac{\partial R}{\partial t} + A \frac{\partial R}{\partial x}
                    + BR\biggr)V + R^{-1}C = 0.\]
  其中 $A' = R^{-1}AR$. 由于
  \[\det(A'-\lambda I) = \det(R^{-1}AR - \lambda I)
    = \det(A - \lambda I),\]
  故方程的根保持不变, 特征方向不变, 特征线也不变.
\end{proof}
% \chapter{广义函数}
\section{基本空间}
1.\textit{Proof}:(反证法)若存在$x_0\in[a,b],s.t.f(x_0)\neq0$,不妨设$f(x_0)>0$,则由连续性知
\[\exists\delta>0,s.t.f(x)>0,\forall x\in(x_0-\delta,x_0+\delta)\cap[a,b]\]
考虑$[a,b]$内紧集$[x_0-\delta/2,x_0+\delta/2]\cap[a,b]$,知存在$\varphi\in C_0^{\infty}([a,b])$,使得$\varphi\geq0$且$\varphi$在该紧集上取1,在$(x_0-\delta,x_0+\delta)$之外取0.将$\varphi$限制在$(a,b)$上知$\varphi\in C_0^{\infty}((a,b))$,但是
\[\int f(x)\varphi(x)\diff x\geq\int_{x_0-\delta/2}^{x_0+\delta/2}f(x)\varphi(x)\diff x>0\]
矛盾,故假设不成立,故$f(x)\equiv0$\\
4.\textit{Proof}:因为
\[\begin{split}
\int|f_{\epsilon}(x)-f(x)|\diff x
&=\int\left|\int(f(x-\tau)-f(x))\varphi_{\epsilon}(\tau)\diff\tau\right|\diff x\\
&\leq\int\left|\int|f(x-\tau)-f(x)||\varphi_{\epsilon}(\tau)|\diff\tau\right|\diff x\\
&=\iint|f(x-\tau)-f(x)|\diff x|\varphi_{\epsilon}(\tau)|\diff\tau\\
&=\iint|f(x-\epsilon s)-f(x)|\diff x\frac{|\varphi(s)|}{\|\varphi\|_{L_1}}\diff s
\end{split}\]
又
\[\int|f(x-\epsilon s)-f(x)|\diff x\leq 2\|f\|_{L_1}\]
\[\lim_{\epsilon\to0}\int|f(x-\epsilon s)-f(x)|\diff x=0\]
由勒贝格控制收敛定理得$\lim_{\epsilon\to0}\|f_{\epsilon}(x)-f(x)\|_{L_1}=0$\\\\
5.\textit{Proof}:\\
以$y$代$x-ty$,则
\[\begin{split}
\partial_{x_i}(t^ku(x,t))
&=\partial_{x_i}\left(t^{k-1}\frac{1}{t^n}\int v(y)\varphi\left(\frac{x-y}{t}\right)\diff y\right)\\
&=t^{k-1}\int v(y)\partial_{x_i}\varphi\left(\frac{x-y}{t}\right)\diff y\\
&=t^{k-1}\int v(x-ty)\partial_{x_i}\varphi(y)\diff y
\end{split}\]
\[\begin{split}
\partial_t(t^ku(x,t))&=\partial_t\left(t^{k-1-n}\int v(y)\varphi\left(\frac{x-y}{t}\right)\diff y\right)\\
&=t^{k-1}\int v(x-ty)((k-n)\varphi(y)-\sum_iy_i\partial_{y_i}\varphi(y))\diff y
\end{split}\]
\newline
6.\textit{Proof}:因为
\[\partial^{\alpha}f_{\epsilon}(x)=\int\partial_x^{\alpha}\frac{1}{c\epsilon^n}\varphi\left(\frac{x-y}{t}\right)\diff y=\frac{1}{c\epsilon^n}\frac{1}{\epsilon^{|\alpha|}}\int\varphi^{(\alpha)}\left(\frac{x-y}{t}\right)\diff y\]
所以
\[|\partial^{\alpha}f_{\epsilon(x)}|\leq\frac{1}{c\epsilon^n}\frac{1}{\epsilon^{|\alpha|}}\int\left|\varphi^{(\alpha)}\left(\frac{x-y}{t}\right)\right|\diff y\]
可以命
\[C(\alpha,n)=\frac{1}{c\epsilon^n}\int\left|\varphi^{(\alpha)}\left(\frac{x-y}{t}\right)\right|\diff y=\frac{1}{c}\int\left|\varphi^{|\alpha|}{\xi}\right|\diff\xi\]
\section{广义函数及其基本运算}
1.\textit{Proof}:$\forall\varphi\in\mathscr{D}(\Omega)$,有
\[
\left\langle\frac{\partial^2u}{\partial x_i\partial x_j},\varphi\right\rangle=-\left\langle\frac{\partial u}{\partial x_i},\frac{\partial\varphi}{\partial x_j}\right\rangle
=\left\langle u,\frac{\partial^2\varphi}{\partial x_j\partial x_i}\right\rangle
=-\left\langle\frac{\partial u}{\partial x_j},\frac{\partial\varphi}{\partial x_i}\right\rangle
=\left\langle\frac{\partial^2u}{\partial x_j\partial x_i},\varphi\right\rangle
\]
故\[\frac{\partial^2u}{\partial x_i\partial x_j}=\frac{\partial^2u}{\partial x_j\partial x_i}\]
\newline
2.\textit{Proof}:$\forall\varphi\in\mathscr{D}(\Omega)$,有
\[\left\langle\frac{\partial}{\partial x_i}(au),\varphi\right\rangle=-\left\langle au,\frac{\partial\varphi}{\partial x_i}\right\rangle=-\left\langle u,a\frac{\partial\varphi}{\partial x_i}\right\rangle=\left\langle u,\varphi\frac{\partial a}{\partial x_i}-\frac{\partial(a\varphi)}{\partial x_i}\right\rangle\]
因为
\[-\left\langle u,\frac{\partial(a\varphi)}{\partial x_i}\right\rangle=\left\langle\frac{\partial u}{\partial x_i},a\varphi\right\rangle=\left\langle a\frac{\partial u}{\partial x_i},\varphi\right\rangle\]
所以
\[\left\langle\frac{\partial}{\partial x_i}(au),\varphi\right\rangle=\left\langle\frac{\partial a}{\partial x_i}u+a\frac{\partial u}{\partial x_i},\varphi\right\rangle\]
故
\[\frac{\partial}{\partial x_i}(au)=a\frac{\partial u}{\partial x_i}+\frac{\partial a}{\partial x_i}u\]
\newline
3.\textit{Proof}:\\
(1)$\forall\varphi\in\mathscr{D}(\Omega)$,有
\[\left\langle\check{\delta},\varphi\right\rangle=\left\langle\delta,\check{\varphi}\right\rangle=\check{\varphi}(0)=\varphi(0)=\left\langle\delta,\varphi\right\rangle\]
\[\left\langle\check{c},\varphi\right\rangle=\left\langle c,\check{\varphi}\right\rangle=\int c\varphi(-x)\diff x=\int c\varphi(x)\diff x=\left\langle c,\varphi\right\rangle\]
故$\delta$和$c$都是偶广义函数\\
(2)设$u$为偶广义函数,$\forall\varphi\in\mathscr{D}(\Omega)$,有
\[\left\langle\check{\left(\frac{\partial u}{\partial x_j}\right)},\varphi\right\rangle=\left\langle\frac{\partial u}{\partial x_j},\check{\varphi}\right\rangle=-\left\langle u,\frac{\partial\check{\varphi}}{\partial x_j}\right\rangle=\left\langle u,\frac{\partial\varphi}{\partial x_j}\right\rangle=-\left\langle\frac{\partial u}{\partial x_j},\varphi\right\rangle\]
故$\frac{\partial u}{\partial x_j}$为奇广义函数\\
(3)$\forall\varphi\in\mathscr{D}(\Omega)$,有
\[\left\langle\check{(u+\check{u})},\varphi\right\rangle\]

\bibliographystyle{plain}
\bibliography{ref}

\end{document}